\section{第一章}

\subsection{1.1}
一个质量为 $\mu$ 的粒子在一维势场 $V(x) = \begin{cases} 0, & |x| > a \\ -V_0, & |x| < a \end{cases}$ 中运动,其中 $V_0 > 0$。求基态能量 $E_0$ 满足的方程;求存在且仅存在一个束缚态的条件。

\subsection{1.2}
质量为 $\mu$ 的粒子在势场 $V(x) = - \alpha \delta (x) (\alpha > 0)$ 中运动, 求束缚态能级和相应的波函数.

\subsection{1.3}
质量为 $\mu$ 的粒子处于二维势场
$$V(x)=\begin{cases}
\infty, & x\geqslant a \\
0, & 0<x<a \\
\infty, & x=0 \\
0, & -a<x<0 \\
\infty, & x\leqslant-a
\end{cases}$$
中,求定态能量 $E$ 与波函数 $\psi(x)$。

\subsection{1.4}
求在半壁无限深方势阱 $V(x) = \begin{cases} \infty, & x < 0 \\ 0, & 0 < x < a \\ V_0, & x > a \end{cases}$ 中存在束缚态的条件 $(V_0 > 0)$。

\subsection{1.5}
质量为 $\mu$ 的粒子在一维势场 
$$V(x) = \begin{cases} 
0, & x<0 \\ 
-\alpha \delta (x) + V_0, & x>0 
\end{cases}$$
中运动,其中 $\alpha$ 与 $V_0$ 均为正实数。(1)给出存在束缚态的条件,并给出能量本征值与相应的本征函数;(2)给出粒子处于 $x > 0$ 区中的概率,它是大于1/2,还是小于1/2?为什么?

\subsection{1.6}
一个质量为 $\mu$ 的粒子在一维势场
$$V(x) = 
\begin{cases} 
\infty, & x < 0, x > a \\
\alpha \delta \left( x - (a/2) \right), & 0 < x < a 
\end{cases}$$
中运动,其中 $\alpha$ 与 $a$ 是正的常数。求第一激发态能量,并讨论 $\alpha \to 0$ 时的定态能量。

\subsection{1.7}
在谐振子势的中心附加$\delta$函数势,$V(x) = \frac{1}{2} \mu \omega^2 x^2 + V_0 \delta (x)$。原来谐振子定态解中,哪些仍是解?哪些不再是解?要重新求。

\subsection{1.8}
求质量为 $\mu$ 的粒子在势场 $V(x) = \begin{cases} 
-\frac{a}{x}, & x > 0 \\ 
\infty, & x < 0 
\end{cases}$ 中的束缚态能量与波函数,其中 $a > 0$。

\subsection{1.9}
求一维氢原子定态能量和波函数,$V(x) = -\frac{e^2}{|x|}$。

\subsection{1.10}
粒子在一维势场 $V(x) = V_0 \left( \frac{a}{x} - \frac{x}{a} \right)^2$ 中运动,其中 $V_0, a$ 是正的常数。求定态能量和波函数。

\subsection{1.11}
设粒子的波函数 $\psi(x) = A \left( \frac{x}{a} \right)^n e^{-x/a}$ 是一维势 $V(x)$ 中的粒子能量本征态,其中 $A, a$ 和 $n$ 为常数。当 $x \to \infty$ 时,$V(x) \to 0$。试求势能 $V(x)$ 和粒子能量 $E$。粒子质量为 $\mu$。

\subsection{1.12}
已知一维定态波函数为 $\psi(x) = \begin{cases} 
a^2 - x^2, & |x| < a \\
0, & |x| > a 
\end{cases}$, 且有 $\langle \psi | V | \psi \rangle = 0$。试从一维定态方程出发,求出势位函数 $V(x)$ 与定态能量 $E$。

\subsection{1.13}
质量为 $\mu$ 的粒子在势场 $V(x) = \begin{cases} 
\infty, & x < 0 \\ 
0, & 0 \leq x \leq a \\
V_0, & x > a 
\end{cases}$ 中运动 $(V_0 > 0)$。已知该粒子在此势场中存在一能量为 $E = V_0 / 2$ 的态。试确定此势阱的宽度 $a$。

\subsection{1.14}
质量为 $\mu$ 的粒子在势场 $V(x)$ 中作一维束缚运动,两个能量本征函数为
$$\psi_1(x) = Ae^{-\beta x^2/2}, \quad \psi_2(x) = B(x^2 + bx + c)e^{-\beta x^2/2}, \quad A, B, b, c \text{均为实常数。试确定参数 } b, c \text{的取值,并求这两个态的能量之差 } E_2 - E_1.$$

\subsection{1.15}
质量为 $\mu$ 的粒子在一圆圈(周长为 $L$)上运动,并受到 $\delta$ 函数势 $V(x) = a\delta \left( x - (L/2) \right)$ 的作用。求粒子能级和相应的归一化波函数。

\subsection{1.16}
粒子在势场 $V(x)$ 中作一维运动,$\hat{H}_0 = \frac{\hat{p}^2}{2\mu} + V(x)$,定态能量为 $E_n^{(0)}, n=1,2,\cdots$。求哈密顿量 $\hat{H} = \hat{H}_0 + \frac{\lambda}{\mu} \hat{p}$ 的本征值,$\lambda$ 为参数。

\subsection{1.17}
电荷为 $q$ 质量为 $\mu$ 的点粒子在一维均匀电场 $E$ 中运动,位势为 $V(x) = -qEx$。在 $t = 0$ 时该粒子的坐标与动量平均值分别为 $\langle x \rangle = x_0$ 与 $\langle \hat{p}_x \rangle = 0$。(1)计算时刻的动量平均值 $\langle \hat{p}_x \rangle$;(2)计算时刻的坐标平均值 $\langle x \rangle$;(3)把计算的结果同经典物理的结果比较。

\subsection{1.18}
质量为 $\mu$ 的粒子被约束在半径为 $r$ 的圆周上运动。(1)设立路障,进一步限制粒子在 $0 < \varphi < \varphi_0$ 的一段圆弧上运动,$V(\varphi) = \begin{cases}
0, & 0 < \varphi < \varphi_0 \\
\infty, & \varphi_0 < \varphi < 2\pi
\end{cases}$,求解粒子的本征能量和本征函数; (2)设粒子处于情况(1)的基态,求突然撤去路障后,粒子仍处于最低能量态的概率。

\subsection{1.19}
质量为 $\mu$ 的粒子处于一维谐振子势场 $V_1 = \frac{1}{2}kx^2$ 的基态,某时刻弹性系数 k 突然变为 2k,即势场变为 $V_2 = kx^2$。求此时粒子处于新势场 $V_2$ 的基态的概率,并求此后粒子能量的平均值。

\subsection{1.20}
同上题。当 $k$ 变成 $2k$ 后,经过多少时间 $T$ 再将 $2k$ 变为 $k$,粒子 100\% 回到原来的基态?

\subsection{1.21}
质量为 $\mu$ 的粒子处于一维刚性盒(0~a)的基态。盒子的 $x=a$ 壁突然运动至 $x=2a$ 处,试计算盒子膨胀后粒子仍处于基态的概率。如果盒子的两壁对称地向两边移动,盒子的宽度由 $a$ 变为 $2a$,结果又如何?

\subsection{1.22}
一个粒子处于宽度为 $a$ 的无限深方势阱中的基态。若(1)阱的两壁同时缓慢地由宽度 $a$ 缩小为 $a/2$;(2)阱的两壁同时突然地由宽度 $a$ 缩小为 $a/2$,求粒子留在基态的概率。

\subsection{1.23}
一个粒子处于宽度为 $a$ 的无限深方势阱的基态,能量为 $E_1 = 38eV$。(1)计算第一激发态能量;(2)计算基态粒子对阱壁的平均作用力。

\subsection{1.24}
质量为 $\mu$ 的粒子处于 $0 < x < a$ 的无限深方势阱中。$t = 0$ 时,归一化波函数为 $\psi(x, 0) = \sqrt{\frac{8}{5a}} \left( 1 + \cos \frac{\pi x}{a} \right) \sin \frac{\pi x}{a}$。求 (1) 在后来某一时刻 $t_0$ 的波函数;(2) 在 $t = 0$ 与 $t = t_0$ 时体系的能量;(3) 在 $t_0$ 时粒子处于 $0 < x < a/2$ 内的概率。

\subsection{1.25}
质量为 $\mu$ 的粒子在无限深方势阱 $V(x) = \begin{cases}
\infty, & x < 0, x > a \\
0, & 0 < x < a
\end{cases}$ 中运动。$t = 0$ 时粒子处于状态 $\psi(x, 0) = A \sin \frac{\pi x}{2a} \cos^3 \frac{\pi x}{2a}$,其中 $A$ 为常数。求出 $t$ 时刻粒子,(1) 处于基态的概率;(2) 能量平均值;(3) 动量平均值;(4) 动量均方差根值(不确定度)。

\subsection{1.26}
质量为 $m$ 的粒子位于一维无限深方势阱 $[-a/2,a/2]$ 中,势阱宽度为 $a$。$t=0$ 时体系处于无限深方势阱中能量最低的两个态的线性叠加态,各自的概率为50\%。计算 $t>0$ 时粒子的概率密度和动量平均值。

\subsection{1.27}
已知 $t = 0$ 时一维运动粒子在态 $\psi (x)$ 中坐标 $x$ 和动量 $\hat{p}$ 的平均值分别为 $x_0$ 与 $p_0$,求 $t = 0$ 时在态 $\varphi (x) = e^{-ip_0 x/\hbar} \psi (x + x_0)$ 中 $x$ 与 $\hat{p}$ 的平均值。

\subsection{1.28}
粒子处于宽度为 $a$ 的一维无限深方势阱 $V(x) = \begin{cases} 
0, & 0 < x < a \\ 
\infty, & x < 0, x > a 
\end{cases}$ 中的定态 $\psi_n(x)$,求粒子的动量分布概率 $|\varphi(p)|^2$。

\subsection{1.29}
粒子处于宽度为 $a$ 的一维无限深方势阱的基态,$t = 0$ 时阱的两壁突然崩溃。求 $t > 0$ 时粒子处于动量取值在 $p \sim p + dp$ 内的概率,以及粒子波函数的表示式(不必算出结果)。

\subsection{1.30}
质量为 $\mu$ 的粒子在一维势场 $V(x) = \begin{cases}
0, & |x| < a \\
\infty, & |x| > a
\end{cases}$ 中运动。(1)求粒子定态能量 $E_n$ 与归一化定态波函数 $\psi_n(x)$。(2)求粒子在定态 $\psi_n(x)$ 上的平均值 $\bar{x}$。(3)设 $t = 0$ 时粒子波函数为 $\psi(x,0) = \begin{cases}
A(a^2 - x^2), & |x| < a \\
0, & |x| > a
\end{cases}$,其中 $A$ 为归一化常数,求 $(a)$ 在 $\psi(x,0)$ 态上粒子能量取值 $E_n$ 的概率;$(b)$ 粒子的平均能量 $\bar{E}$;$(c)$ 任意 $t > 0$ 时粒子波函数 $\psi(x,t)$ 表示式。

\subsection{1.31}
设一维运动粒子的坐标和动量分别为 $q$ 和 $\hat{p}$。(1)计算 $\hat{p}$ 和 $e^{icq}$ 的对易关系,其中 $c$ 为常数;(2)若 $p_0$ 是 $\hat{p}$ 的本征值,相应的本征函数是 $\psi_0(q)$,证明 $p_0 + \hbar c$ 也是 $\hat{p}$ 的本征值,给出相应的本征函数。

\subsection{1.32}
写出二维谐振子势 $V = \frac{1}{2} \mu \omega_x^2 x^2 + \frac{1}{2} \mu \omega_y^2 y^2$ 中粒子能级:(1)设 $\omega_x = \omega_y$,求能级简并度;(2)设 $\omega_x / \omega_y = 1/2$,求能级简并度。

\subsection{1.33}
粒子在二维势场 $V(x, y) = \frac{1}{2} \mu \omega^2 (x^2 + y^2 + 2 \lambda xy)$ 中运动, 其中 $|\lambda| < 1,  \mu$ 为粒子质量. 求能量本征值和本征函数.

解答:
由于

得到哈密顿量


2.令
\begin{equation}
    \xi =\frac{1}{\sqrt{2}}\left( x+y \right) ,\quad \eta =\frac{1}{\sqrt{2}}\left( x-y \right) 
\end{equation}
即
\begin{equation}
    x=\frac{1}{\sqrt{2}}\left( \xi +\eta \right)  ,\quad y=\frac{1}{\sqrt{2}}\left( \xi -\eta \right) 
\end{equation}
得到
\begin{equation}
    \begin{aligned}
        xy&=\frac{1}{\sqrt{2}}\left( \xi +\eta \right) \frac{1}{\sqrt{2}}\left( \xi -\eta \right) 
\\
&=\frac{1}{2}\left( \xi ^2-\eta ^2 \right) 
    \end{aligned}
\end{equation}
且
\begin{equation}
    \begin{aligned}
        x^2+y^2&=\frac{1}{2}\left( \xi +\eta \right) ^2+\frac{1}{2}\left( \xi -\eta \right) ^2
\\
&=\frac{1}{2}\left( \xi ^2+\eta ^2+2\xi \eta \right) 
\\
&=\xi ^2+\eta ^2
    \end{aligned}
\end{equation}
同时
\begin{equation}
    \frac{\partial \xi}{\partial x}=\frac{1}{\sqrt{2}},\quad \frac{\partial \xi}{\partial y}=\frac{1}{\sqrt{2}},\quad \frac{\partial \eta}{\partial x}=\frac{1}{\sqrt{2}},\quad \frac{\partial \eta}{\partial y}=-\frac{1}{\sqrt{2}}
\end{equation}
可得到
\begin{equation}
    \begin{aligned}
        \frac{\partial ^2}{\partial x^2}&=\frac{\partial}{\partial x}\frac{\partial}{\partial x}
\\
&=\left( \frac{\partial \xi}{\partial x}\frac{\partial}{\partial \xi}+\frac{\partial \eta}{\partial x}\frac{\partial}{\partial \eta} \right) \left( \frac{\partial \xi}{\partial x}\frac{\partial}{\partial \xi}+\frac{\partial \eta}{\partial x}\frac{\partial}{\partial \eta} \right) 
\\
&=\left( \frac{1}{\sqrt{2}}\frac{\partial}{\partial \xi}+\frac{1}{\sqrt{2}}\frac{\partial}{\partial \eta} \right) \left( \frac{1}{\sqrt{2}}\frac{\partial}{\partial \xi}+\frac{1}{\sqrt{2}}\frac{\partial}{\partial \eta} \right) 
\\
&=\frac{1}{2}\left( \frac{\partial}{\partial \xi}\frac{\partial}{\partial \xi}+\frac{\partial}{\partial \eta}\frac{\partial}{\partial \xi}+\frac{\partial}{\partial \xi}\frac{\partial}{\partial \eta}+\frac{\partial}{\partial \eta}\frac{\partial}{\partial \eta} \right) 
\\
&=\frac{1}{2}\left( \frac{\partial ^2}{\partial \xi ^2}+\frac{\partial ^2}{\partial \eta ^2}+2\frac{\partial ^2}{\partial \xi \partial \eta} \right) 
    \end{aligned}
\end{equation}
且同样
\begin{equation}
    \begin{aligned}
        \frac{\partial ^2}{\partial y^2}&=\frac{\partial}{\partial y}\frac{\partial}{\partial y}
\\
&=\left( \frac{\partial \xi}{\partial y}\frac{\partial}{\partial \xi}+\frac{\partial \eta}{\partial y}\frac{\partial}{\partial \eta} \right) \left( \frac{\partial \xi}{\partial y}\frac{\partial}{\partial \xi}+\frac{\partial \eta}{\partial y}\frac{\partial}{\partial \eta} \right) 
\\
&=\left( \frac{1}{\sqrt{2}}\frac{\partial}{\partial \xi}-\frac{1}{\sqrt{2}}\frac{\partial}{\partial \eta} \right) \left( \frac{1}{\sqrt{2}}\frac{\partial}{\partial \xi}-\frac{1}{\sqrt{2}}\frac{\partial}{\partial \eta} \right) 
\\
&=\frac{1}{2}\left( \frac{\partial}{\partial \xi}\frac{\partial}{\partial \xi}-\frac{\partial}{\partial \eta}\frac{\partial}{\partial \xi}-\frac{\partial}{\partial \xi}\frac{\partial}{\partial \eta}+\frac{\partial}{\partial \eta}\frac{\partial}{\partial \eta} \right) 
\\
&=\frac{1}{2}\left( \frac{\partial ^2}{\partial \xi ^2}+\frac{\partial ^2}{\partial \eta ^2}-2\frac{\partial ^2}{\partial \xi \partial \eta} \right) 
    \end{aligned}
\end{equation}
得到
\begin{equation}
    \begin{aligned}
        \frac{\partial ^2}{\partial x^2}+\frac{\partial ^2}{\partial y^2}&=\frac{1}{2}\left( \frac{\partial ^2}{\partial \xi ^2}+\frac{\partial ^2}{\partial \eta ^2}+2\frac{\partial ^2}{\partial \xi \partial \eta} \right) +\frac{1}{2}\left( \frac{\partial ^2}{\partial \xi ^2}+\frac{\partial ^2}{\partial \eta ^2}-2\frac{\partial ^2}{\partial \xi \partial \eta} \right) 
\\
&=\frac{1}{2}\left( \frac{\partial ^2}{\partial \xi ^2}+\frac{\partial ^2}{\partial \eta ^2}+2\frac{\partial ^2}{\partial \xi \partial \eta}+\frac{\partial ^2}{\partial \xi ^2}+\frac{\partial ^2}{\partial \eta ^2}-2\frac{\partial ^2}{\partial \xi \partial \eta} \right) 
\\
&=\frac{\partial ^2}{\partial \xi ^2}+\frac{\partial ^2}{\partial \eta ^2}
    \end{aligned}
\end{equation}
将结论
代入
得到
\begin{equation}
    \begin{aligned}
        \hat{H}&=-\frac{\hbar ^2}{2\mu}\left( \frac{\partial ^2}{\partial \xi ^2}+\frac{\partial ^2}{\partial \eta ^2} \right) +\frac{1}{2}\mu \omega ^2\left( \xi ^2+\eta ^2+\lambda \left( \xi ^2-\eta ^2 \right) \right) 
\\
&=-\frac{\hbar ^2}{2\mu}\left( \frac{\partial ^2}{\partial \xi ^2}+\frac{\partial ^2}{\partial \eta ^2} \right) +\frac{1}{2}\mu \omega ^2\left[ \left( 1+\lambda \right) \xi ^2+\left( 1-\lambda \right) \eta ^2 \right] 
\\
&=-\frac{\hbar ^2}{2\mu}\left( \frac{\partial ^2}{\partial \xi ^2}+\frac{\partial ^2}{\partial \eta ^2} \right) +\frac{1}{2}\mu \omega ^2\left( 1+\lambda \right) \xi ^2+\frac{1}{2}\mu \omega ^2\left( 1-\lambda \right) \eta ^2
    \end{aligned}
\end{equation}
令
\begin{equation}
    \omega _{1}^{2}=\omega ^2\left( 1+\lambda \right) ,\quad \omega _{2}^{2}=\omega ^2\left( 1-\lambda \right) 
\end{equation}


\subsection{1.34}
两个质量都是 $\mu$ 的一维耦合谐振子体系的哈密顿量为
$$\hat{H} = \frac{1}{2\mu} (\hat{p}_1^2 + \hat{p}_2^2) + \frac{1}{2} \mu \omega^2 [(x_1 - a)^2 + (x_2 + a)^2 + \lambda (x_1 - x_2)^2]$$
其中 $\lambda$ 与 $a$ 为参数,$\lambda > -1/2$, $- \infty < x_1, x_2 < \infty$. 求体系能量本征值。

\subsection{1.35}
质量为 $\mu$ 的粒子在势场
$$V(x, y, z) = A(x^2+y^2+2\lambda xy)+B(z^2+2cz)$$
中运动,$A, B > 0, |\lambda| < 1$, $c$ 取任意实数,求能量本征值。如考虑另一个新势 $V'$,它同原势的关系为
$$V' = 
\begin{cases}
V, & z > -c, xy \text{任意}\\
\omega, & z < -c, xy \text{任意}
\end{cases}$$
求基态能量。

\subsection{1.36}
设一维粒子由 $x = -\infty$ 处以平面波 $\psi_{in} = e^{ikx}$ 入射,在原点处受到势能 $V(x) = V_0 \delta (x)$ 的作用。(1)写出波函数的一般表达式;(2)确定粒子波函数在原点处满足的边界条件;(3)求出该粒子的透射系数和反射系数;(4)分别指出 $V_0 > 0$ 与 $V_0 < 0$ 时的量子力学效应。

\subsection{1.37}
在以下两种情况下计算入射粒子在一维阶跃势上的反射率 $R$ 与透射率 $T$,$V(x) = \begin{cases} 
0, & x < 0 \\ 
V_0, & x > 0 
\end{cases}$: (1) $E > V_0$; (2) $E < V_0$。

\subsection{1.38}
电子经1000V电压加速后由 $x = -\infty$ 射向阶跃势 $V(x) = \begin{cases}
0, & x < 0 \\
V_0, & x > 0 
\end{cases}$, 其中 $V_0 = 750eV$. 现有1800个电子入射, 在 $x = \infty$ 处能观测到多少个电子?

\subsection{1.39}
粒子被一维矩形势垒 $V(x) = \begin{cases}
0, & x<0, x>a \\
V_0, & 0<x<a
\end{cases}$ 散射。(1)当粒子的能量 $E > V_0$ 时,求反射率 $R$ 与透射率 $T$;(2)当粒子的能量 $E = V_0$ 时,有一半粒子被反射回去,求粒子的质量所满足的方程。

\subsection{1.40}
把传导电子限制在金属内部的是一种平均势,对于下面的一维势模型:
$$V(x) = 
\begin{cases} 
-V_0, & x < 0 \\ 
0, & x > 0 
\end{cases}$$
试就 (1) $E > 0$;(2) $-V_0 < E < 0$ 两种情况下计算接近金属表面的传导电子的反射率与透射率。

\subsection{1.41}
质量为 $m$ 的电子以动能 $E > V_0$ 由左向右入射到高度为 $V_0$ 的台阶势上。在台阶势的跃起处还存在 $\delta$ 势:$\gamma \delta(x)(y > 0)$,即考虑电子在势 $V(x) = V_0 \theta(x) + \gamma \delta(x), \theta(x) = \begin{cases}
0, & x < 0 \\
1, & x > 0
\end{cases}$ 上的散射,$\theta(x)$ 为单位阶跃函数。(1)列出定态薛定谔方程及波函数导数 $\psi'$ 在 $x = 0$ 两侧的跃变条件;(2)求电子在 $x = 0$ 处的透射系数 $T = \left| \frac{j_{out}}{j_{in}} \right|$ 和反射系数 $R = \left| \frac{j_{ref}}{j_{in}} \right|$。

\subsection{1.42}
求一维常虚势场 $-iV(V \ll E)$ 中运动的粒子波函数,计算概率流密度,并证明虚势代表粒子被吸收,求吸收系数 $M$,用 $V$ 表示。

\subsection{1.43}
质量为 $\mu$ 的粒子在势场 $V(x) = \begin{cases}
\infty, & x < 0 \\
-V_0 a \delta (x-a), & x > 0
\end{cases}$ 中运动,其中 $V_0$ 和 $a$ 都是正实数。求(1)束缚态能量满足的方程;(2)存在束缚态的条件。

\subsection{1.44}
质量为 $\mu$ 的粒子在势场 $V(x) = \begin{cases} 
0, & 0 < x < a \\ 
\infty, & x < 0, x > a 
\end{cases}$ 中运动。$t = 0$ 时粒子的波函数为 $\psi (x, 0) = A \left( 1 + 2b \cos \frac{\pi x}{a} \right) \sin \frac{\pi x}{a}$,其中 $A, b$ 为常数。求任意 $t$ 时粒子的波函数 $\psi (x, t)$,平均能量 $E(t)$ 和平均动量 $p(t)$。

\subsection{1.45}
对于一维谐振子哈密顿量 $\hat{H} = \frac{\hat{p}^2}{2\mu} + \frac{1}{2}\mu\omega^2 x^2$,求海森伯绘景中的坐标 $\hat{x}(t)$ 与动量 $\hat{p}(t)$。

\subsection{1.46}
若在薛定谔绘景中,体系的哈密顿量 $\hat{H} = \omega \hat{L}_x$,试给出在海森伯绘景中的 $\hat{L}_x(t)$ 与 $\hat{L}_y(t)$。

\subsection{1.47}
设一维体系能量算符 $\hat{H} = \frac{1}{2 \mu} \left( \hat{p}^2 - \frac{\hbar^2}{x^2} \right)$。(1)利用维里定理证明该体系不存在束缚态;(2)用海森伯运动方程证明算符 $\hat{Q}(t)=\frac{1}{4}(\hat{x}\hat{p}+\hat{p}x)-\hat{H}t$ 在任意态上的平均值为常数。

\subsection{1.48}
什么是量子化?如何实现量子化?