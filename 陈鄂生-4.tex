\section{4}

\subsection{4.1}
质量为 $\mu$ 的粒子在三维球方势阱
$$ V(r) = 
\begin{cases} 
0, & r > a \\ 
-V_0, & r < a 
\end{cases} \quad (V_0 > 0) $$
中运动。求存在 $s$ 波束缚态的条件。

\subsection{4.2}
粒子处于三维球壳势阱$V(r) = -g \delta (r - a)$中,求存在束缚态的条件。

\subsection{4.3}
质量为 $\mu$ 的粒子在势场 $$ V(r) = 
\begin{cases} 
0, & a < r < b \\ 
\infty, & r \leq a, r > b 
\end{cases} $$ 中运动,求 $l = 0$ 的定态能量和定态波函数。

\subsection{4.4}
设粒子的定态波函数 $\psi(r) = Ae^{-r/a}$,其中 $A$ 与 $a$ 是常数。已知 $r \to \infty$,$V(r) = 0$。求定态能量 $E$ 和势能 $V(r)$。

\subsection{4.5}
一个质量为 $\mu$ 带有电荷 $q$ 的粒子被限制在 $xy$ 平面内的半径为 $a$ 的圆周上运动。在 z 轴方向加上强度为 $B$ 的均匀磁场,求粒子的基态能量和基态波函数的表达式,证明基态能量是 $B$ 的周期函数,并给出周期来。已知在柱坐标系中
$$ \nabla^2 = \frac{1}{\rho}\frac{\partial}{\partial \rho}\rho \frac{\partial}{\partial \rho} + \frac{1}{\rho^2}\frac{\partial^2}{\partial \varphi^2} + \frac{\partial^2}{\partial z^2} $$

\subsection{4.6}
质量为 $\mu$ 电荷为 $q$ 的粒子在均匀磁场 $B = Bk$ 中运动,求定态能量与波函数。

\subsection{4.7}
质量为 $\mu$ 电荷为 $q$ 的粒子在方向互相垂直的均匀电场 $E$ 和均匀磁场 $B$ 中运动,求定态能量与波函数。

\subsection{4.8}
(1)已知带有电荷 $q$ 的粒子在磁场 $B$ 与势场 $V(r)$ 中运动,求带电粒子速度分量算符的对易关系 $[\hat{v}_x, \hat{v}_y]$, $[\hat{v}_y, \hat{v}_z]$, $[\hat{v}_z, \hat{v}_x]$ 的表达式。(2)质量为 $\mu$ 带有电荷 $q$ 的粒子在磁场中的哈密顿量为 $\hat{H} = \frac{1}{2 \mu} \left( \hat{p} - \frac{q}{c} A \right)^2$。请问在什么条件下,它可以写成如下形式:$\hat{H}' = \frac{1}{2 \mu} \hat{p}^2 - \frac{q}{\mu c} A \cdot \hat{p} + \frac{q^2}{2 \mu c^2} A^2$。(3)设 $A = A_0 \cos \omega t$ ($A_0$ 为常数矢量),略去 $\hat{H}'$ 中的 $A^2$ 项,求出与 $\hat{H}'$ 相应的薛定谔方程的解。

\subsection{4.9}
处于基态的类氢原子经 $\beta$ 衰变,核电荷数突然由 $Z$ 变为 $Z+1$,求原子处于 $2s$ 态的概率。已知类氢原子定态波函数为
$$ \psi_{100}(Z,r) = \frac{1}{\sqrt{\pi}} \left( \frac{Z}{a} \right)^{3/2} e^{-Zr/a}, \quad \psi_{200}(Z,r) = \frac{1}{\sqrt{\pi}} \left( \frac{Z}{2a} \right)^{3/2} \left( 1 - \frac{Zr}{2a} \right) e^{-Zr/2a} $$

\subsection{4.10}
氢原子处于基态。假定库仑作用在 $t=0$ 时突然消失,电子离开原子像自由电子那样运动。试求 $t>0$ 时测量电子动量大小在 $p\sim p+dp$ 内的概率。

\subsection{4.11}
在半径为 $ R $ 的硬刚球内,有一质量为 $ \mu $ 的粒子。(1)求粒子的基态能量和波函数;(2)如 $ t < 0 $ 时粒子处于基态,$ t = 0 $ 时将这硬刚球的半径扩展到原来的 2 倍,求扩展后粒子仍处于基态的概率。

\subsection{4.12}
一粒子被束缚在半径为 $ R $ 的刚球盒内。求处于基态的粒子对盒壁的压力与压强。

\subsection{4.13}
在势场 $ V(r) $ 中粒子处于定态,证明粒子动能 $ \hat{T} = \hat{p}^2 / 2\mu $ 的平均值为
$$ \bar{T} = \frac{1}{2}(r \cdot \nabla V) $$
如果 $ V(r) $ 是 $ r $ 的 $ v $ 次齐次函数,证明 $ \bar{T} = \frac{v}{2} \bar{V} $,并利用此式,计算氢原子基态的 $ \bar{T} $。

\subsection{4.14}
势能 $ V = -\frac{Ze^2}{r} $ 的类氢原子处于 $ \psi_{nlm} $ 态。试计算 $ \frac{1}{r} $ 的平均值 $\left\langle nlm | \frac{1}{r} | nlm \right\rangle$。

\subsection{4.15}
势能 $ V = -\frac{Ze^2}{r} $ 的类氢原子处于 $\psi_{nlm}$ 态。试计算 $\frac{1}{r^2}$ 的平均值 $\left\langle nlm \right| \frac{1}{r^2} |nlm\rangle$。

\subsection{4.16}
设一粒子在中心力场 $ V(r) $ 中运动, 定义径向动量
$$ \hat{p}_r = \frac{1}{2} \left( \frac{r}{r} \cdot \hat{p} + \hat{p} \cdot \frac{r}{r} \right) $$
(1) 证明
$$ \hat{p}_r = -i \hbar \left( \frac{\partial}{\partial r} + \frac{1}{r} \right) $$
$$ \hat{H} = \frac{\hat{p}_r^2}{2\mu} + \frac{\hat{L}^2}{2\mu r^2} + V(r) $$

(2) 计算对易关系式 $\left[ \frac{\partial}{\partial r}, \hat{H} \right]$; (3) 当粒子处于某一束缚定态 $\psi_{nlm}(r) = R_{nl}(r)Y_{lm}(\theta, \varphi)$ 时, 证明
$$ \left\langle \frac{\partial V}{\partial r} \right\rangle_{nlm} - \frac{l(l+1)\hbar^2}{\mu} \left\langle \frac{1}{r^3} \right\rangle_{nlm} = \frac{\hbar^2}{2\mu} |R_{nl}(0)|^2 $$

(4) 设 $V = -\frac{Ze^2}{r}$, 证明在束缚定态 $\psi_{nlm}(r)$ 上,
$$ \left\langle \frac{1}{r^3} \right\rangle_{nlm} = \frac{Z}{l(l+1)a} \left\langle \frac{1}{r^2} \right\rangle_{nlm} $$

\subsection{4.17}
氢原子处于基态 $\psi(r) = \frac{1}{\sqrt{\pi a^3}} e^{-r/a}$, 求 $r$ 的平均值及动量的概率分布函数 $W(p)$.

\subsection{4.18}
氢原子处于基态 $\psi(r) = \frac{1}{\sqrt{\pi a^3}} e^{-r/a}$,计算 $\Delta x \Delta p_x$,检验测不准关系。

\subsection{4.19}
固有长度为 $r_0$ 的平面转子处于状态 $\psi_m(\varphi) = \sqrt{\frac{1}{2\pi}} e^{im\varphi} (m = 0, \pm 1, \pm 2, \cdots)$,其中平面极角 $\varphi$ 与坐标 $x$ 和 $y$ 的关系是 $x = r_0 \cos \varphi,  y = r_0 \sin \varphi$。求坐标 $x$ 和动量 $p_x$ 的不确定关系 $\Delta x \Delta p_x = ?$

\subsection{4.20}
利用测不准关系估算氢原子基态能量。

\subsection{4.21}
利用测不准关系, 估算质量为 $\mu$ 的粒子在如下势场中的基态能量:
(1) $V(r) = kr(k > 0)$; (2) $V(r) = -\frac{\lambda}{r^{3/2}} (\lambda > 0)$.

\subsection{4.22}
原子核的线度为 $10^{-13}$ cm. 试用不确定原理估算核内质子的动能(以电子伏特为单位).

\subsection{4.23}
一个质量为 $\mu$ 的粒子在对数势场 $V(r) = c \ln \frac{r}{r_0}$ 中运动, 其中 $c$ 与 $r_0$ 是同质量 $\mu$ 无关的常数.(1)证明在所有定态上均方速度相同, 求出这个均方速度;(2)证明任何两个定态能量之差同粒子的质量无关.

\subsection{4.24}
两个质量均为 m 的粒子,通过三维球对称势 $V(r) = c \ln (r/r_0) $ ($c > 0, r_0 > 0$) 而束缚在一起,r 为两粒子之间的距离。已知它的第一激发态与基态的能量之差为 $\Delta E$。今有一个质量为 m 的粒子与另一个质量为 1840m 的粒子通过同一位势形成束缚态,求这一系统第一激发态与基态的能量之差。请说明理由,并给以证明。

\subsection{4.25}
设一微观粒子在中心力场 $V(r)$ 中运动,且处于能量和轨道角动量的某一共同本征态。(1)在球坐标系中写出能量本征函数的基本形式,写出势能 $V(r)$ 在此态上的平均值 $\langle V \rangle$ 的表达式,并最后表示成径向积分的形式;(2)设 $V(r)$ 是 r 的单调上升函数(对任意 $ r $,$ dV/dr > 0 $),证明对任意给定的 $ r_0 $,均有
$$ \int_{0}^{r_0} [V(r) - \langle V \rangle] R^2 (r) r^2 dr < 0, $$
其中 $ R(r) $ 是径向波函数。

\subsection{4.26}
在 $ t = 0 $ 时,氢原子的波函数为
$$ \psi(r, 0) = \frac{1}{\sqrt{10}} (2 \psi_{100} + \psi_{210} + \sqrt{2} \psi_{211} + \sqrt{3} \psi_{21-1}) $$
其中下标分别是量子数 $ n, l, m $ 的值,不考虑自旋。(1)求体系的平均能量;(2)在任意 $ t $ 时刻体系处于 $ l = 1, m = 1 $ 的态的概率是多少?(3)在任意 $ t $ 时刻体系处于 $ m = 0 $ 的态的概率是多少?(4)写出任意 $ t $ 时刻体系的波函数 $ \psi(r, t) $。

\subsection{4.27}
质量为 $ \mu $ 电荷为 $ q $ 的粒子在均匀恒定磁场中运动,取不对称规范:
$ A_x = -By, A_y = A_z = 0, $
B 为磁场大小,则可知 $ y_0 = -c p_x /q B $ 是守恒量。证明下面的量:$ x_0 = x + (c \hat{p}_y / qB) $ 也是守恒量,它与 $ y_0 $ 是否可以同时被观测?

\subsection{4.28}
质量为 $\mu$ 的粒子在中心力场 $ V(r) = -\frac{\alpha}{r^s} (\alpha > 0) $ 中运动。证明存在束缚态的条件是 $ 0 < s < 2 $。

\subsection{4.29}
氘核是由质子和中子组成的唯一的束缚态。实验测定氘核的结合能为 2.23MeV。设质子和中子的作用力势可近似表示为 $$ V(r) = 
\begin{cases}
-V_0, & r < a \\
0, & r > a
\end{cases} $$,求作用力程 $ a $ 和作用强度 $ V_0 $ 之间的关系式。

\subsection{4.30}
双原子分子中两原子之间的作用力势可表示为
$$ V(r) = V_0 \left[ 1 - e^{-(r-R)/a} \right]^2 - V_0 $$
其中 $ r $ 为两原子之间的距离,$ R $ 与 $ a $ 为正的常数,$ R > a $,且 $ e^{R/a} \gg 1 $。当 $ r \to 0 $ 时,$ V(r) \to \infty $; 随 $ r $ 增大,$ V(r) $ 迅速变小;当 $ r = R $ 时,$ V(r) $ 取最小值 $-V_0$; 当 $ r \to \infty $ 时,$ V(r) \to 0 $。求轨道角动量 $ l = 0 $ 的束缚定态能量。

\subsection{4.31}
粒子在中心力场 $V(r) = \frac{A}{r^2} - \frac{B}{r}$ 中运动,其中 $A$ 与 $B$ 为正实数。求定态能量和波函数。(提示:将定态方程化为氢原子定态方程)

\subsection{4.32}
粒子在势场 $ V(r) = -V_0 e^{-r/a} $ 中运动,其中 $V_0$ 与 $a$ 为正实数。求存在束缚态的条件。

\subsection{4.33}
设氢原子处于 $\psi(r) = R_{21}(r)Y_{1-1}(\theta,\varphi)$ 态,其中
$$ R_{21}(r) = \frac{1}{2\sqrt{6a^3/2}} \frac{r}{a} e^{-r/2a}, \quad Y_{1-1}(\theta,\varphi) = \sqrt{\frac{3}{8\pi}} \sin\theta e^{-i\varphi} $$
(1) 求势能 $ V = -e^2 / r $ 的平均值 $\langle V \rangle$;(2) 求轨道角动量 $\hat{L}_z \hat{L}_x^2$ 的平均值 $\langle \hat{L}_z \hat{L}_x^2 \rangle$。

\subsection{4.34}
质量为 $\mu$ 电荷为 $q$ 的粒子在三维各向同性谐振子场 $V(r) = \frac{1}{2}\mu \omega^2 r^2$ 中运动,同时受到一个沿 $x$ 方向的均匀电场 $E = E_0 i$ 的作用。求粒子的能量本征值和第一激发态的简并度。此时轨道角动量是否守恒?如回答是,写出此时守恒量的表达式。

\subsection{4.35}
二维平面有一个以原点为圆心,半径为 $R$ 的光滑刚性圆环,上面套着两颗质量为 $m$,电荷为 $q$ 的“量子珍珠”。“量子珍珠”可以无摩擦地绕环滑动,相互之间存在库仑位势 $V = q^2 / r$,$r$ 为两颗珍珠之间的距离,两颗珍珠各自的方位角(或极角)为 $\alpha, \beta$。(1) 利用 $\alpha, \beta$ 写出这一珍珠体系的定态薛定谔方程;(2) 令 $\lambda = (\alpha + \beta)/2, \theta = \alpha - \beta$,证明定态方程可以利用 $\lambda, \theta$ 分离变量,从而这一体系可以分解为关于 $\lambda$ 的“质心”运动和关于 $\theta$ 的相对运动;(3) 求解关于 $\lambda$ 的“质心”运动;(4) 当 $q^2$ 非常大时,求解或估算关于 $\theta$ 的相对运动的基态与最低激发态的本征能量。

\subsection{4.36}
粒子被限制在无限长圆筒内运动,圆筒半径为 $ a $,求粒子的能量。