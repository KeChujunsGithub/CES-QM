\section{2}
请找全文件中的第二章所有习题,识别文件中的所有题目,包括文字和公式,并按题号顺序输出latex代码。
注意输出格式和图片中的一致。
注意行间公式使用$$(公式)$$,行内公式使用$(公式)$
注意只要题干,不要包含解答部分。

上面所有的习题与解答按照以下格式输出
题号:使用%\subsection{题号}
题目:不发生任何变化,原样输出。

\subsection{2.1}
证明空间反演算符 $\hat{\Pi}[\hat{\Pi}\psi(x)=\psi(-x)]$ 是厄米算符。指出在什么条件下,$\hat{p}=-i\hbar\frac{d}{dx}$ 是厄米算符。

\subusbsection{解答}
\begin{equation}
    \begin{aligned}
        \int_{-\infty}^{+\infty}{\mathrm{d}x}\psi ^*(x)\hat{\Pi}\varphi (x)&=\int_{-\infty}^{+\infty}{\mathrm{d}x}\psi ^*(x)\varphi (-x)
\\
&=\int_{{\color[RGB]{240, 0, 0} +\infty }}^{{\color[RGB]{240, 0, 0} -\infty }}{\mathrm{d}\left( -x^{\prime} \right)}\psi ^*(-x^{\prime})\varphi (x^{\prime})
\\
&=-\int_{{\color[RGB]{240, 0, 0} +\infty }}^{{\color[RGB]{240, 0, 0} -\infty }}{\mathrm{d}x^{\prime}}\psi ^*(-x^{\prime})\varphi (x^{\prime})
\\
&=\int_{-\infty}^{+\infty}{\mathrm{d}x}\psi ^*(-x^{\prime})\varphi (x^{\prime})
\\
&=\int_{-\infty}^{+\infty}{\mathrm{d}x}\psi ^*(-x)\varphi (x)
\\
&=\int_{-\infty}^{+\infty}{\mathrm{d}x}\left[ \hat{\Pi}\psi (x) \right] ^*\varphi (x)
    \end{aligned}
\end{equation}
得到
\begin{equation}
    \int_{-\infty}^{+\infty}{\mathrm{d}x}\psi ^*(x)\left[ \hat{\Pi}\psi (x) \right] =\int_{-\infty}^{+\infty}{\mathrm{d}x}\left[ \hat{\Pi}\psi (x) \right] ^*\varphi (x)
\end{equation}

\subsubsection{解答2}


最终得到
\begin{equation}
    \int_{-\infty}^{+\infty}{\mathrm{d}x}\psi ^*(x)\bigl[ \hat{p}\psi (x) \bigr] =\int_{-\infty}^{+\infty}{\mathrm{d}x}\bigl[ \hat{p}\psi (x) \bigr] ^*\varphi (x)
\end{equation}
因此只要波函数满足束缚态条件或周期性边界条件,$\hat{p}$ 就是厄米算符.

\subsection{2.2}
动量在径向方向的分量定义为 $\hat{p}_r = \frac{1}{2} \left( \hat{p} \cdot \frac{r}{r} + \frac{r}{r} \cdot \hat{p} \right)$。求出 $\hat{p}_r$ 在球坐标中的表达式。



\subsection{2.3}
证明 $[\hat{p}_x, f(x)] = -i\hbar \frac{\partial}{\partial x} f(x)$, $[x, f(\hat{p}_x)] = i\hbar \frac{\partial}{\partial \hat{p}_x} f(\hat{p}_x)$.

\newpage
\subsection{2.4}
设算符 $\hat{A}$ 满足条件 $\hat{A}^2 = 1$,证明 $e^{i\alpha \hat{A}} = \cos \alpha + i \sin \alpha \hat{A}$,其中 $\alpha$ 为实常数。

\subsubsection{解答}
1.由
\begin{equation}
    \hat{A}^2=1
\end{equation}
得到
\begin{equation}
    \begin{aligned}
        \hat{A}^{2k}&=(\hat{A}^2)^k=1^k=1
\\
\hat{A}^{2k+1}&=\hat{A}\cdot (\hat{A}^2)^k=\hat{A}\cdot 1^k=\hat{A}
    \end{aligned}
\end{equation}
2.泰勒展开
\begin{equation}
    \begin{aligned}
        e^x&=\sum_{n=0}^{\infty}{\frac{x^n}{n!}}
\\
\cos x&=\sum_{n=0}^{\infty}{\frac{\left( -1 \right) ^n}{\left( 2n \right) !}}x^{2n}
\\
\sin x&=\sum_{n=0}^{\infty}{\frac{\left( -1 \right) ^n}{\left( 2n+1 \right) !}}x^{2n+1}
    \end{aligned}
\end{equation}
3.计算
\begin{equation}
    \begin{aligned}
        e^{i\alpha \hat{A}}&=\sum_{n=0}^{\infty}{\frac{(i\alpha \hat{A})^n}{n!}}
\\
&=\sum_{n=0}^{\infty}{\frac{i^n\alpha ^n}{n!}}\hat{A}^n
\\
&=\sum_{k=0}^{\infty}{\frac{i^{2k}\alpha ^{2k}}{\left( 2k \right) !}}\hat{A}^{2k}+\sum_{k=0}^{\infty}{\frac{i^{2k+1}\alpha ^{2k+1}}{\left( 2k+1 \right) !}}\hat{A}^{2k+1}
\\
&=\sum_{k=0}^{\infty}{\frac{\left( -1 \right) ^k}{\left( 2k \right) !}}\alpha ^{2k}+i\hat{A}\sum_{k=0}^{\infty}{\frac{\left( -1 \right) ^k}{\left( 2k+1 \right) !}}\alpha ^{2k+1}
\\
&=\cos \alpha +i\hat{A}\sin \alpha 
    \end{aligned}
\end{equation}

其中
\begin{equation}
    \begin{aligned}
        \left( i\alpha \right) ^{2k}&=i^{2k}\alpha ^{2k}=\left( i^2 \right) ^k\alpha ^{2k}=\left( -1 \right) ^k\alpha ^{2k}
\\
\left( i\alpha \right) ^{2k+1}&=i^{2k+1}\alpha ^{2k+1}=i\cdot \left( i^2 \right) ^k\alpha ^{2k+1}=i\cdot \left( -1 \right) ^k\alpha ^{2k+1}
    \end{aligned}
\end{equation}

\newpage
\subsection{2.5}
设算符 $\hat{K} = \hat{L} \hat{M}$,$\hat{L} \hat{M} - \hat{M} \hat{L} = 1$,又设 $\varphi$ 为 $\hat{K}$ 的本征矢,相应本征值为 $\lambda$。证明 $u = \hat{L} \varphi$ 和 $v = \hat{M} \varphi$ 也是 $\hat{K}$ 的本征矢,并求出相应的本征值。

\subsubsection{2.5解答}
由于
\begin{equation}
    [ \hat{L},\hat{M} ] =\hat{L}\hat{M}-\hat{M}\hat{L}=1
\end{equation}
得到
\begin{equation}
    \hat{M}\hat{L}=\hat{L}\hat{M}-1,\hat{L}\hat{M}=\hat{M}\hat{L}+1
\end{equation}

由于设 $\varphi$ 为 $\hat{K}$ 的本征矢,相应本征值为 $\lambda$
得到
\begin{equation}
    \hat{K}\varphi =\lambda \varphi 
\end{equation}

对于
\begin{equation}
    \hat{K}u=\mu u
\end{equation}
代入
\begin{equation}
    \begin{aligned}
        \hat{K}u&=\hat{L}\hat{M}\hat{L}\varphi 
\\
&=\hat{L}\left( \hat{L}\hat{M}-1 \right) \varphi 
\\
&=\hat{L}\hat{L}\hat{M}\varphi -\hat{L}\varphi 
\\
&=\hat{L}\lambda \varphi -\hat{L}\varphi 
\\
&=\left( \lambda -1 \right) \hat{L}\varphi 
\\
&=\left( \lambda -1 \right) u
    \end{aligned}
\end{equation}
得到
\begin{equation}
    \hat{K}u=\left( \lambda -1 \right) u
\end{equation}
因此

对于
\begin{equation}
    \hat{K}v=\nu v
\end{equation}
得到
\begin{equation}
    \begin{aligned}
        \hat{K}v&=\hat{L}\hat{M}\hat{M}\varphi 
\\
&=\left( \hat{M}\hat{L}+1 \right) \hat{M}\varphi 
\\
&=\hat{M}\hat{L}\hat{M}\varphi +\hat{M}\varphi 
\\
&=\hat{M}\lambda \varphi +\hat{M}\varphi 
\\
&=\left( \lambda +1 \right) \hat{M}\varphi 
\\
&=\left( \lambda +1 \right) v
    \end{aligned}
\end{equation}
得到
\begin{equation}
    \hat{K}v=\left( \lambda +1 \right) v
\end{equation}
因此

\subsection{2.6}
粒子作一维运动,$\hat{H} = \frac{\hat{p}^2}{2\mu} + V(x)$,定态波函数为 $|n\rangle : \hat{H}|n\rangle = E_n |n\rangle$,$n = 1, 2, \cdots$。
\\(1) 证明
$$\langle n | \hat{p} | m \rangle = a_{nm} \langle n | x | m \rangle \tag{1}$$
并求系数 $a_{nm}$。
\\(2) 利用式(1)推导求和公式
$$\sum_n (E_n - E_m)^2 | \langle n | x | m \rangle |^2 = \frac{\hbar^2}{\mu^2} \langle m | p^2 | m \rangle \tag{2}$$
\\(3) 证明
$$\sum_n (E_n - E_m) | \langle n | x | m \rangle |^2 = \frac{\hbar^2}{2\mu} \tag{3}$$

\subsection{2.7}
设 $\hat{F}$ 为厄米算符,证明在能量表象中下式成立:
$$\sum_{n} (E_n - E_k) |F_{nk}|^2 = \frac{1}{2} \langle k| [\hat{F}, [\hat{H}, \hat{F}]] |k \rangle$$

\subsection{2.8}
有一量子力学体系,哈密顿量 $\hat{H}$ 的本征值与本征矢量分别为 $E_n$ 与 $|n\rangle: \hat{H}|n\rangle = E_n|n\rangle$。设 $\hat{F}$ 为任一算符 $\hat{F} = \hat{F}(x, \hat{p})$, 试证明
$$\langle k|[\hat{F}^+, [\hat{H}, \hat{F}]] |k \rangle = \sum_{n} (E_n - E_k) \left( \langle n|\hat{F}|k\rangle^2 + \langle k|\hat{F}|n\rangle^2 \right)$$

\subsection{2.9}
已知 $Y_{lm}(\theta, \phi)$ 是 $\hat{L}^2$ 和 $\hat{L}_z$ 的共同本征函数,本征值分别为 $l(l+1)\hbar^2$ 和 $m\hbar$。令 $\hat{L}_{\pm} = \hat{L}_x \pm i \hat{L}_y$。(1) 证明 $\hat{L}_{\pm} Y_{lm}(\theta, \phi)$ 仍是 $\hat{L}^2$ 和 $\hat{L}_z$ 的共同本征函数,求出它们的本征值;(2) 推导公式
$$\hat{L}_{\pm} Y_{lm}(\theta, \phi) = \sqrt{l(l+1) - m(m \pm 1)} \hbar Y_{l m \pm 1}(\theta, \phi)$$

\subsubsection{2.9.1}

\subsubsection{2.9.2}


\subsection{2.10}
证明 $e^{\hat{A}} \hat{B} e^{-\hat{A}} = \hat{B} + [\hat{A}, \hat{B}] + \frac{1}{2!} [\hat{A}, [\hat{A}, \hat{B}]] + \frac{1}{3!} [\hat{A}, [\hat{A}, [\hat{A}, \hat{B}]]] + \cdots$。

\subsection{2.11}
设算符 $\hat{A}$ 与 $\hat{B}$ 同它们的对易关系式 $[\hat{A}, \hat{B}]$ 都对易,证明
$$[\hat{A}, \hat{B}^m] = m \hat{B}^{m-1} [\hat{A}, \hat{B}] \tag{1}$$
$$e^{\hat{A} + \hat{B}} = e^{\hat{A}} e^{\hat{B}} e^{-\frac{1}{2} [\hat{A}, \hat{B}]} \quad \text{或} \quad e^{\hat{A}} e^{\hat{B}} = e^{\hat{A} + \hat{B} + \frac{1}{2} [\hat{A}, \hat{B}]} \tag{2}$$

\subsection{2.12}
设 $\hat{L}$ 为轨道角动量算符。已知 $\hat{L}^2$ 与 $\hat{L}_z$ 的共同本征函数为 $Y_{lm} (\theta, \varphi)$。证明 $\mathrm{e}^{\frac{1}{\mathrm{i}\hbar}\hat{L}_y\frac{\pi}{2}}Y_{lm}(\theta ,\varphi )$ 和 $\hat{L}_x$ 的共同本征函数, 并求出相应的本征值。

\subsection{2.13}
设 $\hat{p}$ 为 $x$ 方向的动量算符,满足对易关系 $[x, \hat{p}] = i\hbar$。求 (1) $e^{i\alpha \hat{p}} x e^{-i\alpha \hat{p}} = ?$ (2) $e^{-i\beta x} [x, e^{i\beta x}] = ?$ 其中 $a, b$ 为常数。

\subsection{2.14}
设粒子处于状态 $Y_{lm}(\theta, \varphi)$,求轨道角动量 $x$ 分量及 $y$ 分量平均值 $\bar{L}_x$ 与 $\bar{L}_y$,以及 $(\Delta L_x)^2$ 与 $(\Delta L_y)^2$。

\subsection{2.15}
已知角动量算符的三个分量 $\hat{J}_x, \hat{J}_y, \hat{J}_z$ 满足对易关系 
$$[\hat{J}_x, \hat{J}_y] = i\hbar \hat{J}_z, [\hat{J}_y, \hat{J}_z] = i\hbar \hat{J}_x, [\hat{J}_z, \hat{J}_x] = i\hbar \hat{J}_y$$
定义: $\hat{J}^2 = \hat{J}_x^2 + \hat{J}_y^2 + \hat{J}_z^2$, $\hat{J}_{\pm} = \hat{J}_x \pm i \hat{J}_y$。(1)求对易关系 $[\hat{J}^2, \hat{J}_{\pm}]$, $[\hat{J}_z, \hat{J}_{\pm}]$, $[\hat{J}_+, \hat{J}_-]$;(2)若 $\hat{J}^2$ 和 $\hat{J}_z$ 的共同本征函数为 $\psi_{jm}$, 其中 $j$ 和 $m$ 为相应的量子数, 证明 $\hat{J}_{\pm} \psi_{jm}$ 也是 $\hat{J}^2$ 和 $\hat{J}_z$ 的共同本征函数,并求出相应的本征值。

\subsection{2.16}
设 $\hat{J}_x, \hat{J}_y, \hat{J}_z$ 为角动量算符,$\hat{J}_{\pm} = \hat{J}_x \pm i\hat{J}_y$。算符 $\hat{V}_+$ 与 $\hat{J}_z, \hat{J}_+$ 满足对易关系:$[\hat{J}_+, \hat{V}_+] = 0, [\hat{J}_z, \hat{V}_+] = \hbar \hat{V}_+$。证明 $\hat{V}_+ | j j \rangle = c | j + 1, j + 1 \rangle$,其中 $c$ 为常数,$|jm\rangle$ 为 $\hat{J}^2$ 与 $\hat{J}_z$ 的共同本征函数。

\subsection{2.17}
令 $\hat{p}_+ = \hat{p}_x + i \hat{p}_y, \hat{L}_+ = \hat{L}_x + i \hat{L}_y$。
(1) 计算 $[\hat{L}_x, \hat{p}_+], [\hat{L}_y, \hat{p}_+], [\hat{L}_z, \hat{p}_+]$。证明
$$[\hat{L}^2, \hat{p}_+] = 2 \hbar (\hat{p}_+ \hat{L}_z - \hat{p}_z \hat{L}_+ ) + 2 \hbar^2 \hat{p}_+$$
(2) 已知 $\hat{L}_z \Phi_m = m \hbar \Phi_m$。证明 $\hat{p}_+ \Phi_m$ 仍是 $\hat{L}_z$ 的本征态,并求出它的本征值。(3) 对于自由粒子体系,已知其哈密顿算符 $\hat{H}$ 同 $\hat{L}^2$ 与 $\hat{L}_z$ 有共同的本征函数完备系 $\psi_{klm} = R_{kl}(r) Y_{lm}(\theta, \varphi)$, 相应的本征值分别是 $E_k = \hbar^2 k^2 / 2\mu, l(l+1)\hbar, m\hbar$。试证明 $\hat{p}_+ \psi_{kll}$ 仍是 $\hat{H}$, $\hat{L}^2$ 与 $\hat{L}_z$ 的共同本征函数,并求出相应的本征值。

\subsection{2.18}
设算符 $\hat{H}$ 具有连续本征值,其本征函数 $u(x, \omega)$ 构成正交完备系。求方程
$$(\hat{H} - \omega_0)V(x) = F(x)$$
的解,其中 $F(x)$ 为已知函数,$\omega_0$ 为某个特定的本征值。

\subsection{2.19}
定义平移算符 $U_x(a)$,它对波函数 $\psi(x)$ 的作用是
$$U_x(a)\psi(x) = \psi(x-a)$$
其中 $a$ 为实数。(1)证明 $U_x(a) = e^{-i a \hat{p}_x / \hbar}$;(2)证明 $U_x(a)$ 为幺正算符。

\newpage
\subsection{2.20}
一维谐振子处于定态 $\psi_n$,计算 $\Delta x \Delta p$,检验测不准关系。
补充

\subsubsection{解法1}
1.对于一维谐振子
\begin{equation}
    \begin{aligned}
        x\psi _n&=\frac{1}{\alpha}\left( \sqrt{\frac{n}{2}}\psi _{n-1}+\sqrt{\frac{n+1}{2}}\psi _{n+1} \right) 
\\
\frac{\mathrm{d}\psi _n}{\mathrm{d}x}&=\alpha \left( \sqrt{\frac{n}{2}}\psi _{n-1}-\sqrt{\frac{n+1}{2}}\psi _{n+1} \right) 
    \end{aligned}
\end{equation}
2.计算期望值
坐标
\begin{equation}
    \begin{aligned}
        \bar{x}&=\int{\psi _{n}^{*}\hat{x}\psi _n\mathrm{d}x}
\\
&=\int{\psi _{n}^{*}x\psi _n\mathrm{d}x}
\\
&=\frac{1}{\alpha}\int{\psi _{n}^{*}\left( \sqrt{\frac{n}{2}}\psi _{n-1}+\sqrt{\frac{n+1}{2}}\psi _{n+1} \right) \mathrm{d}x}
\\
&=\frac{1}{\alpha}\sqrt{\frac{n}{2}}\int{\psi _{n}^{*}\psi _{n-1}\mathrm{d}x}+\frac{1}{\alpha}\sqrt{\frac{n+1}{2}}\int{\psi _{n}^{*}\psi _{n+1}\mathrm{d}x}
\\
&=\frac{1}{\alpha}\sqrt{\frac{n}{2}}\cdot 0+\frac{1}{\alpha}\sqrt{\frac{n+1}{2}}\cdot 0
\\
&=0
    \end{aligned}
\end{equation}
动量
\begin{equation}
    \begin{aligned}
        \bar{p}&=-\mathrm{i}\hbar \int{\psi _{n}^{*}\hat{p}\psi _n\mathrm{d}x}
\\
&=-\mathrm{i}\hbar \int{\psi _{n}^{*}\frac{\mathrm{d}\psi _n}{\mathrm{d}x}\mathrm{d}x}
\\
&=-\mathrm{i}\hbar \alpha \int{\psi _{n}^{*}\left( \sqrt{\frac{n}{2}}\psi _{n-1}-\sqrt{\frac{n+1}{2}}\psi _{n+1} \right) \mathrm{d}x}
\\
&=-\mathrm{i}\hbar \alpha \sqrt{\frac{n}{2}}\int{\psi _{n}^{*}\psi _{n-1}\mathrm{d}x}+\mathrm{i}\hbar \alpha \sqrt{\frac{n+1}{2}}\int{\psi _{n}^{*}\psi _{n+1}\mathrm{d}x}
\\
&=-\mathrm{i}\hbar \alpha \sqrt{\frac{n}{2}}\cdot 0+\mathrm{i}\hbar \alpha \sqrt{\frac{n+1}{2}}\cdot 0
\\
&=0
    \end{aligned}
\end{equation}
2.2计算平方的期望值
坐标
\begin{equation}
    \begin{aligned}
        \overline{x^2}&=\int{\psi _{n}^{*}\hat{x}^2\psi _n\mathrm{d}x}
\\
&=\int{\left( \hat{x}\psi _n \right) ^*\hat{x}\psi _n\mathrm{d}x}
\\
&=\int{\left( x\psi _n \right) ^*x\psi _n\mathrm{d}x}
\\
&=\frac{1}{\alpha ^2}\int{\left( \sqrt{\frac{n}{2}}\psi _{n-1}+\sqrt{\frac{n+1}{2}}\psi _{n+1} \right) ^*\left( \sqrt{\frac{n}{2}}\psi _{n-1}+\sqrt{\frac{n+1}{2}}\psi _{n+1} \right) \mathrm{d}x}
\\
&=\frac{1}{\alpha ^2}\int{\left( \sqrt{\frac{n}{2}}\psi _{n-1}^{*}+\sqrt{\frac{n+1}{2}}\psi _{n+1}^{*} \right) \left( \sqrt{\frac{n}{2}}\psi _{n-1}+\sqrt{\frac{n+1}{2}}\psi _{n+1} \right) \mathrm{d}x}
\\
&=\frac{1}{\alpha ^2}\frac{n}{2}\int{\psi _{n-1}^{*}\psi _{n-1}\mathrm{d}x}+\frac{1}{\alpha ^2}\frac{n+1}{2}\int{\psi _{n+1}^{*}\psi _{n+1}\mathrm{d}x}+\frac{1}{\alpha ^2}\frac{\sqrt{n\left( n+1 \right)}}{2}\int{\psi _{n+1}^{*}\psi _{n-1}\mathrm{d}x}+\frac{1}{\alpha ^2}\frac{\sqrt{n\left( n+1 \right)}}{2}\int{\psi _{n-1}^{*}\psi _{n+1}\mathrm{d}x}
\\
&=\frac{1}{\alpha ^2}\frac{n}{2}\cdot 1+\frac{1}{\alpha ^2}\frac{n+1}{2}\cdot 1+\frac{1}{\alpha ^2}\frac{\sqrt{n\left( n+1 \right)}}{2}\cdot 0+\frac{1}{\alpha ^2}\frac{\sqrt{n\left( n+1 \right)}}{2}\cdot 0
\\
&=\frac{1}{\alpha ^2}\left( n+\frac{1}{2} \right) 
    \end{aligned}
\end{equation}
动量
\begin{equation}
    \begin{aligned}
        \overline{p^2}&=\int{\psi _{n}^{*}\hat{p}^2\psi _n\mathrm{d}x}
\\
&=\int{\left( \hat{p}\psi _n \right) ^*\hat{p}\psi _n\mathrm{d}x\mathrm{d}x}
\\
&=\hbar ^2\int{\left( \frac{\mathrm{d}\psi _n}{\mathrm{d}x} \right) ^*\left( \frac{\mathrm{d}\psi _n}{\mathrm{d}x} \right) \mathrm{d}x}
\\
&=\hbar ^2\alpha ^2\int{\left( \sqrt{\frac{n}{2}}\psi _{n-1}-\sqrt{\frac{n+1}{2}}\psi _{n+1} \right) ^*\left( \sqrt{\frac{n}{2}}\psi _{n-1}-\sqrt{\frac{n+1}{2}}\psi _{n+1} \right) \mathrm{d}x}
\\
&=\hbar ^2\alpha ^2\int{\left( \sqrt{\frac{n}{2}}\psi _{n-1}^{*}-\sqrt{\frac{n+1}{2}}\psi _{n+1}^{*} \right) \left( \sqrt{\frac{n}{2}}\psi _{n-1}-\sqrt{\frac{n+1}{2}}\psi _{n+1} \right) \mathrm{d}x}
\\
&=\frac{\hbar ^2\alpha ^2n}{2}\int{\psi _{n-1}^{*}\psi _{n-1}\mathrm{d}x}+\frac{\hbar ^2\alpha ^2\left( n+1 \right)}{2}\int{\psi _{n+1}^{*}\psi _{n+1}\mathrm{d}x}-\frac{\hbar ^2\alpha ^2\sqrt{n\left( n+1 \right)}}{2}\int{\psi _{n+1}^{*}\psi _{n-1}\mathrm{d}x}-\frac{\hbar ^2\alpha ^2\sqrt{n\left( n+1 \right)}}{2}\int{\psi _{n-1}^{*}\psi _{n+1}\mathrm{d}x}
\\
&=\frac{\hbar ^2\alpha ^2n}{2}\cdot 1+\frac{\hbar ^2\alpha ^2\left( n+1 \right)}{2}\cdot 1-\frac{\hbar ^2\alpha ^2\sqrt{n\left( n+1 \right)}}{2}\cdot 0-\frac{\hbar ^2\alpha ^2\sqrt{n\left( n+1 \right)}}{2}\cdot 0
\\
&=\hbar ^2\alpha ^2\biggl( n+\frac{1}{2} \biggr) 
    \end{aligned}
\end{equation}
3.由上面计算的结论
\begin{equation}
    \begin{aligned}
        \bar{x}&=0\Rightarrow \bar{x}^2=0
\\
\bar{p}&=0\Rightarrow \bar{p}^2=0
\\
\overline{x^2}&=\frac{1}{\alpha ^2}\left( n+\frac{1}{2} \right) 
\\
\overline{p^2}&=\hbar ^2\alpha ^2\biggl( n+\frac{1}{2} \biggr) 
    \end{aligned}
\end{equation}
计算方差
\begin{equation}
    \begin{aligned}
        \Delta x&=\sqrt{\overline{x^2}-\bar{x}^2}=\frac{1}{\alpha}\sqrt{n+\frac{1}{2}}
\\
\Delta p&=\sqrt{\overline{p^2}-\bar{p}^2}=\hbar \alpha \sqrt{n+\frac{1}{2}}
    \end{aligned}
\end{equation}
计算不确定关系
\begin{equation}
    \begin{aligned}
        \Delta x\Delta p&=\frac{1}{\alpha}\sqrt{n+\frac{1}{2}}\cdot \hbar \alpha \sqrt{n+\frac{1}{2}}
\\
&=\left( n+\frac{1}{2} \right) \hbar 
    \end{aligned}
\end{equation}
符合
\begin{equation}
    \Delta x\Delta p\geqslant \frac{\hbar}{2}
\end{equation}

\subsubsection{解法2}
1.对于一维谐振子
\begin{equation}
    \begin{aligned}
        x\psi _n&=\frac{1}{\alpha}\left( \sqrt{\frac{n}{2}}\psi _{n-1}+\sqrt{\frac{n+1}{2}}\psi _{n+1} \right) 
\\
\frac{\mathrm{d}\psi _n}{\mathrm{d}x}&=\alpha \left( \sqrt{\frac{n}{2}}\psi _{n-1}-\sqrt{\frac{n+1}{2}}\psi _{n+1} \right) 
    \end{aligned}
\end{equation}
写出
\begin{equation}
    \begin{aligned}
        x\psi _{n-1}&=\frac{1}{\alpha}\left( \sqrt{\frac{n-1}{2}}\psi _{n-2}+\sqrt{\frac{n}{2}}\psi _n \right) 
\\
x\psi _{n+1}&=\frac{1}{\alpha}\left( \sqrt{\frac{n+1}{2}}\psi _n+\sqrt{\frac{n+2}{2}}\psi _{n+2} \right) 
\\
\frac{\mathrm{d}\psi _{n-1}}{\mathrm{d}x}&=\alpha \left( \sqrt{\frac{n-1}{2}}\psi _{n-2}-\sqrt{\frac{n}{2}}\psi _n \right) 
\\
\frac{\mathrm{d}\psi _{n+1}}{\mathrm{d}x}&=\alpha \left( \sqrt{\frac{n+1}{2}}\psi _n-\sqrt{\frac{n+2}{2}}\psi _{n+2} \right) 
    \end{aligned}
\end{equation}
可计算
\begin{equation}
    \begin{aligned}
        x^2\psi _n&=x\cdot x\psi _n
\\
&=x\cdot \frac{1}{\alpha}\left( \sqrt{\frac{n}{2}}\psi _{n-1}+\sqrt{\frac{n+1}{2}}\psi _{n+1} \right) 
\\
&=\frac{1}{\alpha}\left( \sqrt{\frac{n}{2}}x\psi _{n-1}+\sqrt{\frac{n+1}{2}}x\psi _{n+1} \right) 
\\
&=\frac{1}{\alpha}\left[ \sqrt{\frac{n}{2}}\frac{1}{\alpha}\left( \sqrt{\frac{n-1}{2}}\psi _{n-2}+\sqrt{\frac{n}{2}}\psi _n \right) +\sqrt{\frac{n+1}{2}}\frac{1}{\alpha}\left( \sqrt{\frac{n+1}{2}}\psi _n+\sqrt{\frac{n+2}{2}}\psi _{n+2} \right) \right] 
\\
&=\frac{1}{\alpha ^2}\left[ \frac{\sqrt{n\left( n-1 \right)}}{2}\psi _{n-2}+\frac{n}{2}\psi _n+\frac{n+1}{2}\psi _n+\frac{\sqrt{\left( n+1 \right) \left( n+2 \right)}}{2}\psi _{n+2} \right] 
\\
&=\frac{1}{2\alpha ^2}\left[ \sqrt{n\left( n-1 \right)}\psi _{n-2}+\left( 2n+1 \right) \psi _n+\sqrt{\left( n+1 \right) \left( n+2 \right)}\psi _{n+2} \right] 
    \end{aligned}
\end{equation}
以及
\begin{equation}
    \begin{aligned}
        \frac{\mathrm{d}^2\psi _n}{\mathrm{d}x^2}&=\frac{\mathrm{d}}{\mathrm{d}x}\frac{\mathrm{d}\psi _n}{\mathrm{d}x}
\\
&=\frac{\mathrm{d}}{\mathrm{d}x}\left[ \alpha \left( \sqrt{\frac{n}{2}}\psi _{n-1}-\sqrt{\frac{n+1}{2}}\psi _{n+1} \right) \right] 
\\
&=\alpha \left( \sqrt{\frac{n}{2}}\frac{\mathrm{d}\psi _{n-1}}{\mathrm{d}x}-\sqrt{\frac{n+1}{2}}\frac{\mathrm{d}\psi _{n+1}}{\mathrm{d}x} \right) 
\\
&=\alpha \left[ \sqrt{\frac{n}{2}}\alpha \left( \sqrt{\frac{n-1}{2}}\psi _{n-2}-\sqrt{\frac{n}{2}}\psi _n \right) -\sqrt{\frac{n+1}{2}}\alpha \left( \sqrt{\frac{n+1}{2}}\psi _n-\sqrt{\frac{n+2}{2}}\psi _{n+2} \right) \right] 
\\
&=\alpha ^2\left[ \frac{\sqrt{n\left( n-1 \right)}}{2}\psi _{n-2}-\frac{n}{2}\psi _n-\frac{n+1}{2}\psi _n+\frac{\sqrt{\left( n+1 \right) \left( n+2 \right)}}{2}\psi _{n+2} \right] 
\\
&=\frac{\alpha ^2}{2}\left[ \sqrt{n\left( n-1 \right)}\psi _{n-2}-\left( 2n+1 \right) \psi _n+\sqrt{\left( n+1 \right) \left( n+2 \right)}\psi _{n+2} \right] 
    \end{aligned}
\end{equation}
2.计算期望值
坐标
\begin{equation}
    \begin{aligned}
        \bar{x}&=\int{\psi _{n}^{*}\hat{x}\psi _n\mathrm{d}x}
\\
&=\int{\psi _{n}^{*}x\psi _n\mathrm{d}x}
\\
&=\frac{1}{\alpha}\int{\psi _{n}^{*}\left( \sqrt{\frac{n}{2}}\psi _{n-1}+\sqrt{\frac{n+1}{2}}\psi _{n+1} \right) \mathrm{d}x}
\\
&=\frac{1}{\alpha}\sqrt{\frac{n}{2}}\int{\psi _{n}^{*}\psi _{n-1}\mathrm{d}x}+\frac{1}{\alpha}\sqrt{\frac{n+1}{2}}\int{\psi _{n}^{*}\psi _{n+1}\mathrm{d}x}
\\
&=\frac{1}{\alpha}\sqrt{\frac{n}{2}}\cdot 0+\frac{1}{\alpha}\sqrt{\frac{n+1}{2}}\cdot 0
\\
&=0
    \end{aligned}
\end{equation}
动量
\begin{equation}
    \begin{aligned}
        \bar{p}&=-\mathrm{i}\hbar \int{\psi _{n}^{*}\hat{p}\psi _n\mathrm{d}x}
\\
&=-\mathrm{i}\hbar \int{\psi _{n}^{*}\frac{\mathrm{d}\psi _n}{\mathrm{d}x}\mathrm{d}x}
\\
&=-\mathrm{i}\hbar \alpha \int{\psi _{n}^{*}\left( \sqrt{\frac{n}{2}}\psi _{n-1}-\sqrt{\frac{n+1}{2}}\psi _{n+1} \right) \mathrm{d}x}
\\
&=-\mathrm{i}\hbar \alpha \sqrt{\frac{n}{2}}\int{\psi _{n}^{*}\psi _{n-1}\mathrm{d}x}+\mathrm{i}\hbar \alpha \sqrt{\frac{n+1}{2}}\int{\psi _{n}^{*}\psi _{n+1}\mathrm{d}x}
\\
&=-\mathrm{i}\hbar \alpha \sqrt{\frac{n}{2}}\cdot 0+\mathrm{i}\hbar \alpha \sqrt{\frac{n+1}{2}}\cdot 0
\\
&=0
    \end{aligned}
\end{equation}
2.2计算
坐标
\begin{equation}
    \begin{aligned}
        \overline{x^2}&=\int{\psi _{n}^{*}\hat{x}^2\psi _n\mathrm{d}x}
\\
&=\int{\psi _{n}^{*}x^2\psi _n\mathrm{d}x}
\\
&=\frac{1}{2\alpha ^2}\int{\psi _{n}^{*}\left[ \sqrt{n\left( n-1 \right)}\psi _{n-2}+\left( 2n+1 \right) \psi _n+\sqrt{\left( n+1 \right) \left( n+2 \right)}\psi _{n+2} \right] \mathrm{d}x}
\\
&=\frac{1}{2\alpha ^2}\left[ \sqrt{n\left( n-1 \right)}\int{\psi _{n}^{*}\psi _{n-2}\mathrm{d}x}+\left( 2n+1 \right) \int{\psi _{n}^{*}\psi _n\mathrm{d}x}+\sqrt{\left( n+1 \right) \left( n+2 \right)}\int{\psi _{n}^{*}\psi _{n+2}\mathrm{d}x} \right] 
\\
&=\frac{1}{2\alpha ^2}\left[ \sqrt{n\left( n-1 \right)}\cdot 0+\left( 2n+1 \right) \cdot 1+\sqrt{\left( n+1 \right) \left( n+2 \right)}\cdot 0 \right] 
\\
&=\frac{1}{2\alpha ^2}\left( 2n+1 \right) 
\\
&=\frac{1}{\alpha ^2}\left( n+\frac{1}{2} \right) 
    \end{aligned}
\end{equation}
动量
\begin{equation}
    \begin{aligned}
        \overline{p^2}&=\int{\psi _{n}^{*}\hat{p}^2\psi _n\mathrm{d}x}
\\
&=-\hbar ^2\int{\psi _{n}^{*}\frac{\mathrm{d}^2\psi _n}{\mathrm{d}x^2}\mathrm{d}x}
\\
&=-\frac{\hbar ^2\alpha ^2}{2}\int{\psi _{n}^{*}\left[ \sqrt{n\left( n-1 \right)}\psi _{n-2}-\left( 2n+1 \right) \psi _n+\sqrt{\left( n+1 \right) \left( n+2 \right)}\psi _{n+2} \right] \mathrm{d}x}
\\
&=-\frac{\hbar ^2\alpha ^2}{2}\left[ \sqrt{n\left( n-1 \right)}\int{\psi _{n}^{*}\psi _{n-2}\mathrm{d}x}-\left( 2n+1 \right) \int{\psi _{n}^{*}\psi _n\mathrm{d}x}+\sqrt{\left( n+1 \right) \left( n+2 \right)}\int{\psi _{n}^{*}\psi _{n+2}\mathrm{d}x} \right] 
\\
&=-\frac{\hbar ^2\alpha ^2}{2}\left[ \sqrt{n\left( n-1 \right)}\cdot 0-\left( 2n+1 \right) \cdot 1+\sqrt{\left( n+1 \right) \left( n+2 \right)}\cdot 0 \right] 
\\
&=\frac{\hbar ^2\alpha ^2}{2}\left( 2n+1 \right) 
\\
&=\hbar ^2\alpha ^2\biggl( n+\frac{1}{2} \biggr) 
    \end{aligned}
\end{equation}
3.由上面计算的结论
\begin{equation}
    \begin{aligned}
        \bar{x}&=0\Rightarrow \bar{x}^2=0
\\
\bar{p}&=0\Rightarrow \bar{p}^2=0
\\
\overline{x^2}&=\frac{1}{\alpha ^2}\left( n+\frac{1}{2} \right) 
\\
\overline{p^2}&=\hbar ^2\alpha ^2\biggl( n+\frac{1}{2} \biggr) 
    \end{aligned}
\end{equation}
计算方差
\begin{equation}
    \begin{aligned}
        \Delta x&=\sqrt{\overline{x^2}-\bar{x}^2}=\frac{1}{\alpha}\sqrt{n+\frac{1}{2}}
\\
\Delta p&=\sqrt{\overline{p^2}-\bar{p}^2}=\hbar \alpha \sqrt{n+\frac{1}{2}}
    \end{aligned}
\end{equation}
计算不确定关系
\begin{equation}
    \begin{aligned}
        \Delta x\Delta p&=\frac{1}{\alpha}\sqrt{n+\frac{1}{2}}\cdot \hbar \alpha \sqrt{n+\frac{1}{2}}
\\
&=\left( n+\frac{1}{2} \right) \hbar 
    \end{aligned}
\end{equation}




\newpage
\subsection{2.21}
$t = 0$ 时一维自由运动粒子的归一化波函数为
$$\psi (x, 0) = (2\pi a^2)^{-1/4} \exp \left[ i k_0 (x - x_0) - \frac{(x - x_0)^2}{(2a)^2} \right]$$
其中 $a, k_0$ 与 $x_0$ 均为正实数。(1) 求 $t = 0$ 时粒子的坐标概率分布函数与坐标分布宽度 $\Delta x = \sqrt{\overline{x}^2 - (\overline{x})^2}$;(2) 求 $t = 0$ 时粒子的动量概率分布函数与动量分布宽度 $\Delta p = \sqrt{\overline{p}^2 - (\overline{p})^2}$,并检验坐标与动量的测不准关系;(3) 求 $t > 0$ 时粒子的波函数 $\psi (x, t)$,坐标 $x$ 的平均值 $\overline{x}(t)$ 及坐标的概率分布函数;(4) 求 $t > 0$ 时粒子动量的平均值 $\overline{p}(t)$ 及动量的概率分布函数。

\subsection{2.22}
已知束缚态波函数为 $\psi(x)$,求动量 $p$ 与动能 $T = p^2 / 2\mu$ 的概率分布函数的表达式。对一维谐振子基态,波函数为
$$\psi(x) = \sqrt{\frac{\alpha}{\pi}} e^{-\alpha^2 x^2 / 2}, \quad \alpha = \sqrt{\mu \omega / \hbar}$$
算出动量 $p$ 与动能 $T$ 的概率分布函数,并算出动能平均值。

\subsection{2.23}
一维谐振子能量的本征值与本征函数为
$$E_n = \left( n + \frac{1}{2} \right) \hbar \omega, \quad \psi_n(x) = N_n e^{-\alpha^2 x^2 / 2} H_n (\alpha x)$$
$$N_n = \sqrt{\frac{\alpha}{\sqrt{\pi} 2^n n!}}, \quad \alpha = \sqrt{\frac{\mu \omega}{\hbar}}, \quad n = 0, 1, 2, \cdots$$
(1) 由厄米多项式 $H_n(z)$ 的递推关系
$$zH_n(z) = \frac{1}{2} H_{n+1}(z) + nH_{n-1}(z)$$
$$H_n'(z) = 2nH_{n-1}(z)$$
导出 $x \psi_n(x)$ 和 $d \psi_n(x) / dx$ 满足的递推关系。(2) 求出 $\psi_n(x)$ 态上坐标和动量的平均值 $\bar{x}$ 和 $\bar{p}$。(3) 证明谐振子零点能 $E_0 = \frac{\hbar \omega}{2}$ 是测不准关系 $\Delta x \Delta p > \frac{\hbar}{2}$ 的直接结果。(4) 求出 $\psi_n(x)$ 态上动能与势能的平均值 $\bar{T}$ 和 $\bar{V}$, 并找出它们之间的关系。

\subsection{2.24}
质量为 $\mu$ 的粒子在外场作用下作一维运动 ($-\infty < x < +\infty$)。已知当其处于束缚态 $\psi_1(x)$ 时, 动能平均值等于 $E_1$, 并已知 $\psi_1(x)$ 是实函数。试求当粒子处于态 $\psi_2(x) = \psi_1(x) e^{ikx}$ ($k$ 为实数)时动量平均值 $\overline{p}$ 与动能平均值 $\overline{T}$。

\subsection{2.25}
一维谐振子哈密顿算符(取 $\hbar = \mu = \omega = 1$)为
$$\hat{H} = \frac{1}{2}(x^2 + \hat{p}^2)$$
其本征值与本征函数为 $E_n$ 与 $\psi_n(x)$。已知 $\psi_n(x)$ 为实函数, 字称为(-1)”。请写出 $E_n$ 的具体形式。已知 t=0 时谐振子的波函数为
$$\psi(0) = \frac{1}{2} \left( \sqrt{2}\psi_0 + \psi_1 + \psi_2 \right)$$
求任意 t 时刻 x 与 p 的平均值 $\overline{x(t)}$ 与 $\overline{p(t)}$。

\subsection{2.26}
证明在宽度为 $a$ 的一维无限深方势阱中的定态能量 $E > \hbar^2 / 2 \mu a^2$。

\subsection{2.27}
(1) 如厄米算符 $\hat{A}$ 对任何态矢量 $|u\rangle$,有 $\langle u|\hat{A}|u\rangle > 0$,则称 $\hat{A}$ 是正定算符。求证算符 $\hat{A} = |a\rangle \langle a|$ 是厄米正定算符。(2) 如 $\hat{A}$ 是任一线性算符,证明 $\hat{A}^{\dagger} \hat{A}$ 是厄米正定算符,它的迹等于 $\hat{A}$ 在任意表象中的矩阵元的模平方之和,试推导,当且仅当 $\hat{A} = 0$ 时,tr$(\hat{A}^{\dagger} \hat{A}) = 0$ 才成立。

\subsection{2.28}
设归一化波函数 $|\psi\rangle$ 满足薛定谔方程 $i \hbar \frac{\partial}{\partial t} |\psi\rangle = \hat{H} |\psi\rangle$。定义密度算符(矩阵)为 $\rho = |\psi\rangle \langle \psi|$。(1) 证明任意力学量 $\hat{F}$ 在 $|\psi\rangle$ 态中的平均值可表示为 tr$(\hat{F}\rho)$;(2) 求出 $\rho$ 的本征值;(3) 导出 $\rho$ 随时间变化的方程。

\subsection{2.29}
已知一量子体系,除了能量之外还包括另外三个可观察量 $\hat{P}, \hat{Q}$ 与 $\hat{R}$。设该体系只有两个能量本征态 $|1\rangle$ 与 $|2\rangle$,它们不一定是 $\hat{P}, \hat{Q}$ 与 $\hat{R}$ 的本征态。基于以下各组“实验数据”,尽可能多地定出 $\hat{P}, \hat{Q}$ 与 $\hat{R}$ 的本征值(有一组数据是非物理的):
(1) $\langle 1|\hat{P}|1\rangle = 1/2, \quad \langle 1|\hat{P}^2|1\rangle = 1/4$;
(2) $\langle 1|\hat{Q}|1\rangle = 1/2, \quad \langle 1|\hat{Q}^2|1\rangle = 1/6$;
(3) $\langle 1|\hat{R}|1\rangle = 1, \quad \langle 1|\hat{R}^2|1\rangle = 5/4, \quad \langle 1|\hat{R}^3|1\rangle = 7/4$。

\subsection{2.30}
已知一体系哈密顿量 $\hat{H}$ 不含 $t$, $\hat{H}$ 具有非简并本征值 $E_{\lambda}, \lambda = 1, 2, \cdots$, 相应本征态矢为 $|\psi_{\lambda}\rangle: \hat{H}|\psi_{\lambda}\rangle = E_{\lambda}|\psi_{\lambda}\rangle$。又已知另一可观察量 $\hat{A}$ 的非简并本征值与本征态矢为 $a_n$ 与 $|\phi_n\rangle: \hat{A}|\phi_n\rangle = a_n|\phi_n\rangle, n = 1, 2, \cdots$。设体系初态为 $|\psi_{\lambda}\rangle$, 问(1)在初态 $|\psi_{\lambda}\rangle$ 下对 $\hat{A}$ 测量时, 测得 $\hat{A}$ 的平均值是什么? 这一测量给出 $\hat{A}$ 的值为 $a_m$ 的概率有多大? (2)如果上次对 $\hat{A}$ 测量得 $a_m$, 经过时间间隔 $t$ 以后再次测量 $\hat{A}$ 仍得 $a_m$ 的概率有多大?

\subsection{2.31}
计算 $\hat{p} \times \hat{L} + \hat{L} \times \hat{p} = ?$

\subsection{2.32}
定义角动量升降算符 $\hat{J}_{\pm} = \hat{J}_x \pm i \hat{J}_y$,(1) 证明算符 $\hat{J}_+ \hat{J}_- $ 与 $\hat{J}_- \hat{J}_+ $ 的厄米性,并求出它们的本征态与本征值;(2) 若力学量算符 $\hat{F}$ 满足对易关系 $[\hat{F}, \hat{J}_{\mu}] = 0, \mu = x, y, z$,试证明 $\hat{F}$ 在 $\hat{J}^2, \hat{J}_z$ 共同本征态上的平均值与磁量子数无关。

\subsection{2.33}
证明力学量 $\hat{A}$(不显含t)的平均值对时间的二次微商为
$$\frac{d^2\overline{A}}{dt^2} = -\frac{1}{\hbar^2} \left\langle \left[ \hat{A}, \hat{H} \right], \hat{H} \right\rangle$$
其中 $\hat{H}$ 为该体系的哈密顿量。

\subsection{2.34}
证明在一维束缚态问题中,位能在基态的平均值满足如下关系:
$$\langle V \rangle_0 = E_0 - \frac{m}{2\hbar^2} \sum_n (E_n - E_0)^2 |\langle n| x | 0 \rangle|^2$$
再证明 $\langle V \rangle_0 \leq (5E_0 - E_1)/4, \quad E_0 \leq E_1 \leq E_2 \leq \cdots$

\subsection{2.35}
能量为 $E$ 的一束粒子穿过小孔,投射到距离小孔距离为 $L$ 的屏上。用不确定原理,证明通过不断减小孔径来达到减少屏上斑点直径是不可行的。估算使屏上斑点直径最小时小孔的直径。

\subsection{2.36}
设体系的能量本征方程为 $\hat{H} | n \rangle = E_n | n \rangle, \langle m | n \rangle = \delta_{mn}, \quad E_0 < E_1 < E_2 \cdots$。
(1) 取 $| \psi_0 \rangle$ 为归一化基态试探态矢。令 $E = \langle \psi_0 | \hat{H} | \psi_0 \rangle, \quad \varepsilon = 1 - | \langle 0 | \psi_0 \rangle |^2$,证明 $E - E_0 > (E_1 - E_0) \varepsilon$。(2) 若只知 $\hat{H}$ 最低的两个本征态矢 $| 0 \rangle$ 与 $| 1 \rangle$,试从任意归一化态矢出发,构造第二激发态的试探态矢,并求出该激发态能量上限。

\subsection{2.37}
设轨道角动量 $\hat{L}$ 在 $n$ 方向上的分量为
$$\hat{L}_n = \hat{L} \cdot n = \hat{L}_x \sin \alpha \cos \beta + \hat{L}_y \sin \alpha \sin \beta + \hat{L}_z \cos \alpha$$
其中 $\alpha, \beta$ 为已知的方位角。求在算符 $\hat{L}^2$ 与 $\hat{L}_z$ 的共同本征态 $| lm \rangle$ 上算符 $\hat{L}_n$ 和 $\hat{L}_n^2$ 的平均值。

\subsection{2.38}
证明在 $\hat{H} = \frac{\hat{p}^2}{2\mu} + V(x)$ 的束缚定态 $\psi_n (x)$ 上,动量 $\hat{p}$ 与作用力 $\hat{F}$ 的平均值为0。

\subsection{2.39}
设体系的哈密顿量 $\hat{H}$ 同力学量 $\hat{A}$ 满足反对易关系 $\hat{H} \hat{A} + \hat{A} \hat{H} = 0$。设 $\psi$ 是 $\hat{H}$ 的本征值为 $E(\neq 0)$ 的本征态,(1) 证明 $\hat{A} \psi$ 是 $\hat{H}$ 的本征值为 $-E$ 的本征态;(2) 求 $\hat{A}$ 在 $\psi$ 态上的平均值。

\subsection{2.40}
已知力学量 $\hat{A}$ 与 $\hat{B}$ 的本征值分别为 $a_n$ 与 $b_n$。在 $|\psi\rangle$ 态上先测量 $\hat{A}$ 得 $a_n$,后测量 $\hat{B}$ 得 $b_n$ 的概率为 $P(a_n, b_n)$;先测量 $\hat{B}$ 得 $b_n$,后测量 $\hat{A}$ 得 $a_n$ 的概率为 $P(b_n, a_n)$。问 $P(a_n, b_n) = P(b_n, a_n)$ 的条件是什么?

\subsection{2.41}
已知可观察量 $A$ 的算符 $\hat{A}$ 有两个本征函数 $\phi_1, \phi_2$, 本征值分别为 $a_1, a_2$; 观察量 $B$ 的算符 $\hat{B}$ 有两个本征函数 $\chi_1, \chi_2$, 本征值分别为 $b_1, b_2$。两种本征态有如下关系:
$$\phi_1 = \frac{2\chi_1 + 3\chi_2}{\sqrt{13}}, \quad \phi_2 = \frac{3\chi_1 - 2\chi_2}{\sqrt{13}}$$
当测量 $\hat{A}$ 后得到 $a_1$, 若再测量 $\hat{B}$, 然后再测量 $\hat{A}$, 问第二次测量 $\hat{A}$ 得到 $a_1$ 的概率是多少?

\subsection{2.42}
设能量 $E$ 是三度简并的, 对应的 3 个波函数为 $\phi_1, \phi_2, \phi_3$, 它们不归一, 相互之间也不正交。试通过它们, 构造出 3 个相互正交、且归一的波函数。

\subsection{2.43}
一体系由两个不同类型的中微子组成,其哈密顿量 $\hat{H}$ 的本征态矢为 $|1\rangle$ 与 $|2\rangle$,相应的本征能量为 $E_1$ 与 $E_2$,$E_1 < E_2$。已知电子中微子与 $\mu$ 子中微子的态矢分别为
$$|e\rangle = \cos \theta |1\rangle + \sin \theta |2\rangle, \quad |\mu\rangle = -\sin \theta |1\rangle + \cos \theta |2\rangle$$
其中 $\theta$ 是混合角。设体系在 $t=0$ 时处于电子中微子态 $|e\rangle$,求(1)任意时刻体系的态矢 $|\psi(t)\rangle$; (2)任意时刻体系处于基态 $|1\rangle$ 的概率; (3)任意时刻体系处于 $\mu$ 子中微子态 $|\mu\rangle$ 的概率; (4)何时体系又回到电子中微子态 $|e\rangle$,周期是什么?

\subsection{2.44}
一维谐振子降算符 $a$ 与升算符 $a^+$ 的定义为
$$a = \sqrt{\frac{\mu\omega}{2\hbar}} \left( x + \frac{i}{\mu\omega} \hat{p} \right), \quad a^+ = \sqrt{\frac{\mu\omega}{2\hbar}} \left( x - \frac{i}{\mu\omega} \hat{p} \right)$$
由 $a$ 与 $a^+$ 可以构成厄米算符 $\hat{N} = a^+ a$。令 $\hat{N}$ 的本征值为 $n$,本征态为 $|n\rangle$。(1) 计算对易关系:$[a,a^+]$, $[a,\hat{N}]$, $[a^+,\hat{N}]$;(2) 证明
$$a|n\rangle = \sqrt{n}|n-1\rangle$$
$$a^+|n\rangle = \sqrt{n+1}|n+1\rangle$$
(3) 将谐振子的哈密顿算符 $\hat{H} = \frac{\hat{p}^2}{2\mu} + \frac{1}{2}\mu\omega^2 x^2$ 用 $\hat{N} = a^+ a$ 表示;(4) 利用式(2)求出 $\hat{N}$ 的本征值 $n$,从而求出 $\hat{H}$ 的本征值 $E$;(5) 由式(3)不难得到
$$|n\rangle = \frac{1}{\sqrt{n!}}(a^+)^n |0\rangle$$
写出它在 $x$ 表象中的表达式 $\psi_n(x) = \langle x|n\rangle$。利用公式(2),给出 $\psi_0(x) = \langle x|0\rangle$ 满足的方程,并求出 $\psi_0(x)$。

\subsection{2.45}
一维谐振子降算符 $a$ 的本征值为 $\alpha$ 的本征态 $|\alpha\rangle$ 称为谐振子的相干态。设谐振子能量本征态为 $|n\rangle$, 由公式
$$a|n\rangle = \sqrt{n}|n-1\rangle$$
看出, 谐振子的基态 $|0\rangle$ 是 $a$ 的本征值 $\alpha=0$ 的本征态。 $a$ 的本征值 $\alpha \neq 0$ 的本征态可以表示成 $|n\rangle$ 的线性叠加:
$$|\alpha\rangle = \sum_{n=0}^{\infty}c_n|n\rangle$$
求出 $c_n$, 从而得到谐振子的相干态 $|\alpha\rangle$。

\subsection{2.46}
设 $|n\rangle$ 为一维谐振子的能量本征态, 谐振子降算符 $a$ 对 $|n\rangle$ 的作用为
$$a|n\rangle = \sqrt{n}|n-1\rangle$$
证明(1) 一维谐振子相干态
$$|\alpha\rangle = e^{-|\alpha|^2/2} \sum_{n=0}^{\infty} \frac{\alpha^n}{\sqrt{n!}} |n\rangle$$
是谐振子降算符 $a$ 的本征态, 本征值为 $\alpha$。(2) $a$ 的不同本征值 $\beta$ 与 $\alpha$ 的本征态不正交
$$\langle \beta | \alpha \rangle \neq 0$$
(3) 谐振子相干态(2)可以表示为
$$|\alpha\rangle = e^{\alpha a^+ - \alpha^* a} |0\rangle$$
其中 $a^+$ 为谐振子升算符。

\subsection{2.47}
一维谐振子处于相干态 $|\alpha\rangle = e^{-|\alpha|^2/2} \sum_{n=0}^{\infty} \frac{\alpha^n}{\sqrt{n!}} |n\rangle$,其中 $|n\rangle$ 是谐振子哈密顿量 $\hat{H} = \frac{\hat{p}^2}{2\mu} + \frac{1}{2}\mu\omega^2 x^2$的本征态。谐振子相干态 $|\alpha\rangle$ 是谐振子算符
$$a = \sqrt{\frac{\mu\omega}{2\hbar}} \left( x + \frac{i}{\mu\omega} \hat{p} \right)$$
的本征值为 $\alpha$的本征态,满足归一化条件 $\langle \alpha | \alpha \rangle = 1$。(1)求能量平均值 $\overline{H}$;(2)求 $\Delta x$与$\Delta p$,表明在相干态上乘积 $\Delta x \Delta p$ 取测不准关系式 $\Delta x \Delta p > \hbar/2$中的最小值。

解答

由
\begin{equation}
    \hat{a}=\sqrt{\frac{\mu \omega}{2\hbar}}\left( \hat{x}+\frac{\mathrm{i}}{\mu \omega}\hat{p} \right) ,\quad \hat{a}^{\dagger}=\sqrt{\frac{\mu \omega}{2\hbar}}\left( \hat{x}-\frac{\mathrm{i}}{\mu \omega}\hat{p} \right) 
\end{equation}
得到
\begin{equation}
    \hat{x}=\sqrt{\frac{\hbar}{2\mu \omega}}\left( \hat{a}^{\dagger}+\hat{a} \right) ,\quad \hat{p}=\mathrm{i}\sqrt{\frac{\mu \omega \hbar}{2}}\left( \hat{a}^{\dagger}-\hat{a} \right) 
\end{equation}
且
\begin{equation}
    \left[ \hat{a},\hat{a}^{\dagger} \right] =\hat{a}\hat{a}^{\dagger}-\hat{a}^{\dagger}\hat{a}=1\Rightarrow \hat{a}\hat{a}^{\dagger}=\hat{a}^{\dagger}\hat{a}+1
\end{equation}
2.计算期望值和平方的期望值
\begin{equation}
    \begin{aligned}
        \bar{x}=\sqrt{\frac{\hbar}{2\mu \omega}}\langle \alpha |\left( \hat{a}^{\dagger}+\hat{a} \right) |\alpha \rangle 
\\
=\sqrt{\frac{\hbar}{2\mu \omega}}\left( \langle \alpha |\hat{a}^{\dagger}|\alpha \rangle +\langle \alpha |\hat{a}|\alpha \rangle \right) 
\\
=\sqrt{\frac{\hbar}{2\mu \omega}}\left( \langle \alpha |\alpha ^*|\alpha \rangle +\langle \alpha |\alpha |\alpha \rangle \right) 
\\
=\sqrt{\frac{\hbar}{2\mu \omega}}\left( \alpha ^*\langle \alpha |\alpha \rangle +\alpha \langle \alpha |\alpha \rangle \right) 
\\
=\sqrt{\frac{\hbar}{2\mu \omega}}\left( \alpha ^*+\alpha \right) 
    \end{aligned}
\end{equation}
和
\begin{equation}
    \begin{aligned}
        \bar{p}&=\mathrm{i}\sqrt{\frac{\mu \omega \hbar}{2}}\langle \alpha |\left( \hat{a}^{\dagger}-\hat{a} \right) |\alpha \rangle 
\\
&=\mathrm{i}\sqrt{\frac{\mu \omega \hbar}{2}}\left( \langle \alpha |\hat{a}^{\dagger}|\alpha \rangle -\langle \alpha |\hat{a}|\alpha \rangle \right) 
\\
&=\mathrm{i}\sqrt{\frac{\mu \omega \hbar}{2}}\left( \langle \alpha |\alpha ^*|\alpha \rangle -\langle \alpha |\alpha |\alpha \rangle \right) 
\\
&=\mathrm{i}\sqrt{\frac{\mu \omega \hbar}{2}}\left( \alpha ^*\langle \alpha |\alpha \rangle -\alpha \langle \alpha |\alpha \rangle \right) 
\\
&=\mathrm{i}\sqrt{\frac{\mu \omega \hbar}{2}}\left( \alpha ^*-\alpha \right) 
    \end{aligned}
\end{equation}
3.计算平方的平均值
\begin{equation}
    \begin{aligned}
        \overline{x^2}&=\frac{\hbar}{2\mu \omega}\langle \alpha |\left( {\hat{a}^{\dagger}}^2+\hat{a}^2+2\hat{a}^{\dagger}\hat{a}+1 \right) |\alpha \rangle 
\\
&=\frac{\hbar}{2\mu \omega}\left( \langle \alpha |{\hat{a}^{\dagger}}^2|\alpha \rangle +\langle \alpha |\hat{a}^2|\alpha \rangle +2\langle \alpha |\hat{a}^{\dagger}\hat{a}|\alpha \rangle +1 \right) 
\\
&=\frac{\hbar}{2\mu \omega}\left( \langle \alpha |{\alpha ^*}^2|\alpha \rangle +\langle \alpha |\alpha ^2|\alpha \rangle +2\langle \alpha |\alpha ^*\alpha |\alpha \rangle +1 \right) 
\\
&=\frac{\hbar}{2\mu \omega}\left( {\alpha ^*}^2\langle \alpha |\alpha \rangle +\alpha ^2\langle \alpha |\alpha \rangle +2\alpha ^*\alpha \langle \alpha |\alpha \rangle +1 \right) 
\\
&=\frac{\hbar}{2\mu \omega}\left( {\alpha ^*}^2+\alpha ^2+2\alpha ^*\alpha +1 \right) 
    \end{aligned}
\end{equation}
且
\begin{equation}
    \begin{aligned}
        \overline{p^2}&=\frac{\mu \omega \hbar}{2}\langle \alpha |\left( 2\hat{a}^{\dagger}\hat{a}+1-{\hat{a}^{\dagger}}^2-\hat{a}^2 \right) |\alpha \rangle 
\\
&=\frac{\mu \omega \hbar}{2}\left( 2\langle \alpha |\hat{a}^{\dagger}\hat{a}|\alpha \rangle +1-\langle \alpha |{\hat{a}^{\dagger}}^2|\alpha \rangle -\langle \alpha |\hat{a}^2|\alpha \rangle \right) 
\\
&=\frac{\mu \omega \hbar}{2}\left( 2\langle \alpha |\alpha ^*\alpha |\alpha \rangle +1-\langle \alpha |{\alpha ^*}^2|\alpha \rangle -\langle \alpha |\alpha ^2|\alpha \rangle \right) 
\\
&=\frac{\mu \omega \hbar}{2}\left( 2\alpha ^*\alpha \langle \alpha |\alpha \rangle +1-{\alpha ^*}^2\langle \alpha |\alpha \rangle -\alpha ^2\langle \alpha |\alpha \rangle \right) 
\\
&=\frac{\mu \omega \hbar}{2}\left( 2\alpha ^*\alpha +1-{\alpha ^*}^2-\alpha ^2 \right) 
    \end{aligned}
\end{equation}

4.计算不确定关系
由
\begin{equation}
    \begin{aligned}
        \bar{x}&=\sqrt{\frac{\hbar}{2\mu \omega}}\left( \alpha ^*+\alpha \right) \Rightarrow \bar{x}^2=\frac{\hbar}{2\mu \omega}\left( \alpha ^{*2}+2\alpha ^*\alpha +\alpha ^2 \right) 
\\
\overline{x^2}&=\frac{\hbar}{2\mu \omega}\left( {\alpha ^*}^2+\alpha ^2+2\alpha ^*\alpha +1 \right) 
    \end{aligned}
\end{equation}
得到
\begin{equation}
    \Delta x=\sqrt{\overline{x^2}-\bar{x}^2}=\sqrt{\frac{\hbar}{2\mu \omega}}
\end{equation}

由
\begin{equation}
    \begin{aligned}
        \bar{p}&=\mathrm{i}\sqrt{\frac{\mu \omega \hbar}{2}}\left( \alpha ^*-\alpha \right) \Rightarrow \bar{p}^2=-\frac{\mu \omega \hbar}{2}\left( \alpha ^{*2}+2\alpha ^*\alpha +\alpha ^2 \right) 
\\
\overline{p^2}&=\frac{\mu \omega \hbar}{2}\left( 2\alpha ^*\alpha +1-{\alpha ^*}^2-\alpha ^2 \right) 
    \end{aligned}
\end{equation}
得到
\begin{equation}
    \Delta p=\sqrt{\overline{p^2}-\bar{p}^2}=\sqrt{\frac{\mu \omega \hbar}{2}}
\end{equation}

5.计算不确定关系
\begin{equation}
    \Delta x\Delta p=\sqrt{\frac{\hbar}{2\mu \omega}}\cdot \sqrt{\frac{\mu \omega \hbar}{2}}=\frac{\hbar}{2}
\end{equation}

\subsection{2.48}
设厄米电荷算符 $\hat{Q}$ 的本征态为 $|\psi_q\rangle$, 本征值为 $q: \hat{Q} |\psi_q\rangle = q |\psi_q\rangle$。电荷共轭算符 $\hat{C}$ 对 $|\psi_q\rangle$ 的作用是使之成为 $\hat{Q}$ 的本征值为 $-q$ 的本征态 $|\psi_{-q}\rangle$:
$$\hat{C}|\psi_q\rangle = |\psi_{-q}\rangle$$
证明算符 $\hat{C}$ 与 $\hat{Q}$ 反对易:$\hat{C}\hat{Q}+\hat{Q}\hat{C}=0$。










