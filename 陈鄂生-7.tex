\section{第七章}

\subsection{7.1}
一体系由三个全同玻色子组成,不考虑粒子间的相互作用。已知可能的单粒子态为 $\varphi_1$ 与 $\varphi_2$,相应的能量为 $\varepsilon_1$ 与 $\varepsilon_2$,写出体系所有可能态的波函数和能量。

\subsection{7.2}
两个自旋 $s = 1/2$ 的全同粒子在同一谐振子势场中运动,$$V(r) = \frac{1}{2} \mu \omega^2 r^2$$,不考虑两粒子之间的相互作用,求一粒子处于单粒子基态,另一粒子处于在$z$方向运动的单粒子第一激发态的体系波函数和能量,并求体系总角动量量子数$(j,m_j)$的可能值。

\subsection{7.3}
(1) 在一维无限深方势阱 $$V(x) = \begin{cases} 0, & 0 < x < a \\ \infty, & x < 0, x > a \end{cases}$$ 中有两个自旋 $s = 0$ 的全同粒子,粒子之间不存在相互作用,写出体系最低两个能级,指出简并度,并给出相应的波函数;(2) 同(1),粒子具有自旋 $s = 1/2$;(3) 同(2),但粒子之间存在同自旋有关的相互作用力势 $V = A S_1 \cdot S_2 (A>0)$。

\subsection{7.4}
设绝对零度时,在三维各向同性谐振子势 $$V(r) = \frac{1}{2}\mu\omega^2r^2$$ 中有 20 个自旋 $s=1/2$ 质量为 $\mu$ 的全同粒子组成的体系。忽略粒子之间的相互作用。已知这 20 个粒子的平均能量为 3eV。(1)如果同样的温度下该势场中有 12 个这样的粒子组成的体系,其平均能量是什么? (2)如果同样的温度下该势场中有 17 个自旋 $s=0$ 质量仍为 $\mu$ 的全同粒子组成的体系,其平均能量是什么?

\subsection{7.5}
设有两个质量为 $\mu$ 自旋 $s=1/2$ 的全同粒子,在同一势场 $V=\frac{1}{2}kx^2$ 中作一维谐振子运动,即 $V_i=\frac{1}{2}kx_i^2(i=1,2)$。两个粒子之间的相互作用力势为 $$\frac{1}{2}\alpha k(x_1-x_2)^2,\alpha>-1/2$$。试求体系的能级,并指出哪些能级属于自旋单态?哪些能级属于自旋三重态?

\subsection{7.6}
假设氢原子中两个电子被没有自旋的玻色子取代(质量和电荷不变),试讨论能级的变化。

\subsection{7.7}
两个质量为 $\mu$ 的粒子处于边长为 $a > b > c$ 的立方体盒中,粒子间相互作用势 $V = A\delta (\tau_1 - \tau_2)$ 可视为微扰。在下列条件下,用一级微扰方法计算体系的最低能量:(1)粒子非全同;(2)零自旋的全同粒子;(3)自旋为1/2的全同粒子,并处于总自旋$s=1$的态上。

\subsection{7.8}
假设两个质量为 $m_q = 70  \text{MeV/c}^2$ 的夸克可以通过相互作用位势 $V(r) = -a(\hat{\sigma}_1 \cdot \hat{\sigma}_2 - b)r^2$ 束缚在一起,其中 $r$ 是两个夸克之间的距离,$b$ 是一个待定的参数,$a = 68.99  \text{MeV/fm}^2$。(1) $b$ 取什么值才能使两个夸克束缚在一起? (2) 设两个夸克是不同类型的,并取 $b = 3/2$。试求基态能量和简并度;(3) 设两个夸克是同一类型的,并取 $b = 3/2$。试求基态能量和简并度;(4) 令 $b = 0$,求两个全同夸克在基态的方均根距离。已知 $hc = 197.3  \text{MeV} \cdot \text{fm}$。

\subsection{7.9}
一体系由两个质量为 $m$ 的一维全同粒子组成,两粒子之间的相互作用为 $a(x_1 - x_2)^2 / 2 (a > 0)$。(1)若粒子自旋为 0,写出它们的相对运动态的能量和波函数;(2)若粒子自旋 $s = 1/2$,写出它们的相对运动态及第一激发态的能量和波函数。

\subsection{7.10}
氯化钠晶体中有些负离子空穴。每个空穴束缚一个电子,可以认为,这些电子被束缚在一个尺度为晶格常数的三维无限深势阱中,晶体处于室温。试粗略估计被这些电子强烈吸收的电磁波的最长波长。已知 $\hbar c = 197MeV \cdot fm$,晶格常数 $a = 0.1nm = 10^5fm$,电子质量 $\mu c^2 = 0.511MeV$。

\subsection{7.11}
设 $a^+$ 与 $a$ 是玻色子在某单粒子态上的产生算符与湮没算符,满足对易关系 $[a, a^+] = 1$。 $\hat{N} = a^+ a$ 是该单粒子态上的粒子占有数算符,$|n\rangle$ 是 $\hat{N}$ 的本征值为 $n$ 的本征态。(1) 计算对易关系 $[a, \hat{N}]$, $[a^+, \hat{N}]$; (2) 证明 $$a | n \rangle = \sqrt{n} | n - 1 \rangle$$ 和 $$a^+ | n \rangle = \sqrt{n + 1} | n + 1 \rangle$$; (3) 求 $\hat{N}$ 的本征值 $n$。

\subsection{7.12}
算符$a_i$与$a_j^+(i,j=1,2)$满足对易关系$[a_i,a_j^+]=\delta_{ij}$,$[a_i,a_j]=0$,$[a_i^+,a_j^+]=0$。体系的哈密顿量为 $$\hat{H}=\hbar\omega_0(a_1^+a_1+a_2^+a_2)+i\hbar\omega_1(a_1^+a_2-a_2^+a_1)$$,其中$\omega_0$与$\omega_1$均为正实数,且$\omega_1 \ll \omega_0$。试用微扰方法计算体系的第一激发态的一级近似能量,并同精确能量比较。

\subsection{7.13}
$a$ 与 $a^+$ 是费米子体系的某个单粒子态的湮没与产生算符,满足反对易关系 $\{a, a^+\} = aa^+ + a^+ a = 1,  a^2 = (a^+) = 0$。以 $\hat{N} = a^+ a$ 表示该单粒子态的粒子占有数算符,$|n\rangle$ 为 $\hat{N}$ 的本征值为 $n$ 的本征态矢。(1)求 $\hat{N}$ 的本征值 $n$;(2)计算对易关系式 $[\hat{N}, a], [\hat{N}, a^+]$;(3)证明 $$a |n\rangle = \sqrt{n}|n-1\rangle,  a^+ |n\rangle = \sqrt{1-n}|n+1\rangle$$;(4)在 $\hat{N}$ 表象求 $a$ 与 $a^+$ 的矩阵表示。

\subsection{7.14}
算符 $a_i$ 与 $a_j^+(i,j=1,2)$ 满足反对易关系 $\{a_i,a_j\}=\{a_i^+,a_j^+\}=0$, $\{a_i,a_j^+\}=a_i a_j^++a_j^+ a_i=\delta_{ij}$。(1)试求哈密顿量 $\hat{H}_0 = \hbar \omega_1 a_1^+ a_1 + \hbar \omega_2 a_2^+ a_2 (\omega_2 > \omega_1 > 0)$ 的能谱和本征态矢; (2)在 $\hat{H}_0$ 表象给出算符 $\hat{Q} = a_1 a_2$ 和 $\hat{W} = a_1^+ a_2^+$ 的矩阵表示; (3)设 $\hat{H}' = \epsilon(a_1^+ a_2^+ - a_1 a_2)$ 为微扰,求 $\hat{H}_0$ 的基态在计入微扰后的二级近似能量和一级近似态矢。

\subsection{7.15}
设 $a$ 和 $a^+$ 是湮没算符和产生算符,满足对易关系 $[a, a^+] = 1$。体系的哈密顿量为 $$\hat{H} = A a^2 + B (a^+) ^2 + C a^+ a + D$$。请问 $A, B, C, D$ 要满足什么条件,$\hat{H}$ 才是厄米算符?求出体系的能量。

\subsection{7.16}
某体系哈密顿量 $$\hat{H} = \frac{5}{3} a^+ a + \frac{2}{3} [a^2 + (a^+)^2]$$,其中 $$a = \frac{1}{\sqrt{2}} (Q + i \hat{P})$$,$$a^+ = \frac{1}{\sqrt{2}} (Q - i \hat{P})$$,$\hat{P}$ 与 $Q$ 满足基本对易关系 $[Q, \hat{P}] = i$,$\hat{P} = -i \frac{d}{dQ}$。试求 $\hat{H}$ 的本征值与基态波函数 $\psi_0(Q)$。

\subsection{7.17}
设哈密顿算符 $\hat{H} = \lambda a^+a + \epsilon(a^++a)$,其中 $\lambda$ 是正实数,$\epsilon$ 是小参数,$a^+$ 与 $a$ 是玻色子产生算符与湮没算符。求 $\hat{H}$ 的基态能量本征值(准至 $\epsilon^2$级),并同严格值比较。

\subsection{7.18}
在粒子数表象中,一维谐振子基态态矢 $|0\rangle$ 满足性质 $a|0\rangle = 0$,其中 $a$ 为湮没算符,$$a = \sqrt{\frac{\mu\omega}{2h}}\left(x + \frac{i}{\mu\omega}\hat{p}\right)$$。试利用此性质求出基态在动量表象中的波函数显示式 $\langle p|0\rangle = \psi_0(p)$。

\subsection{7.19}
一维谐振子哈密顿量 $$\hat{H} = \hbar \omega \left( a^+ a + \frac{1}{2} \right)$$,且 $[a, a^+] = 1$。(1)若 $|0\rangle$ 是归一化的基态态矢 $(a | 0\rangle = 0)$,则第 $n$ 个激发态态矢为 $|n\rangle = N_n (a^+)^n |0\rangle$。求归一化因子 $N_n$。(2)若外加一微扰 $\hat{H}' = ga^+ a^+ a a$,求第 $n$ 个激发态的能量本征值(准至 $g$ 一级)。

\subsection{7.20}
一维谐振子受到微扰 $\hat{H}' = cx^2$ 的作用,其中 $c$ 是常数。在粒子数表象中 $$x = \sqrt{\frac{h}{2\mu\omega}} (a+a^+)$$,$a$ 与 $a^+$ 分别是湮没算符和产生算符,满足如下公式:$$a|n\rangle = \sqrt{n}|n-1\rangle,  a^+|n\rangle = \sqrt{n+1}|n+1\rangle$$,其中 $|n\rangle$ 是一维谐振子哈密顿量 $$\hat{H} = h\omega \left( a^+ a + \frac{1}{2} \right)$$ 的本征态。(1)用微扰论,准确到二级近似,求能量修正值;(2)求能量的准确值,并与微扰论给出的结果比较。

\subsection{7.21}
考虑两个自旋 $s = 1/2$,质量为 $\mu$ 的全同粒子在三维空间内运动。假定两粒子的总动量为0,两粒子间的相互作用势为 $$V(r_1, r_2) = \frac{g}{|r_1 - r_2|} \sigma_1 \cdot \sigma_2$$,其中 $g$ 是一个正实数,$\sigma_1$ 与 $\sigma_2$ 为两个粒子的泡利矩阵。求出两个粒子能形成的所有束缚态能级和相应的简并度。

\subsection{7.22}
两个无相互作用粒子置于二维无限深方势阱 $(0 < x < a)$ 中。对于以下两种情况,写出两粒子体系可具有的两个最低能量值,相应的简并度,以及上述能级对应的所有二粒子波函数:(1)两个自旋为1/2的可区分粒子;(2)两个自旋为1/2的全同粒子。

\subsection{7.23}
两个无相对作用粒子具有相同质量 $m$,在宽为 $a$ 的一维无限深方势阱中运动。(1)写出体系4个最低能级的能量。(2)对下述情况,分别求出体系4个最低能级的简并度:(a)自旋为1/2的全同粒子;(b)自旋为1/2的非全同粒子;(c)自旋为1的全同粒子。

\subsection{7.24}
求两个自旋$s=1/2$的关在一维无限深势阱$$V(x)= \begin{cases} 0, & 0<x<a \\ \infty, & x<0,x>a \end{cases}$$中,并以接触势$U(x_1,x_2)=c\delta(x_1-x_2)(c\ll 1)$为相互作用的全同粒子系统的零级近似归一化波函数(考虑自旋态),以接触势为微扰,求准确到$c$的一次方的基态能量。

\subsection{7.25}
考虑两个具有相同角频率 $a_0$ 的振子,哈密顿量为 $\hat{H}_1 = ha_0 a_1^+ a_1$, $\hat{H}_2 = ha_0 a_2^+ a_2$。记 $\hat{H}_1, \hat{H}_2$ 相应于本征值 $n_1 ha_0$ 和 $n_2 ha_0$ 的本征态为 $|n_1 n_2\rangle$,零点能量已略去。在两个振子具有相互作用后,其哈密顿量为 $$\hat{H} = ha_0 a_1^+ a_1 + ha_0 a_2^+ a_2 + ga_1^+ a_2 + ga_2^+ a_1 = (a_1^+, a_2^+) \begin{pmatrix} ha_0 & g \\ g & ha_0 \end{pmatrix} \begin{pmatrix} a_1 \\ a_2 \end{pmatrix}$$ 其中 $g$ 为正实数。因为有相互作用,$|n_1 n_2\rangle$ 不是 $\hat{H}$ 的本征态。求 $\hat{H}$ 的本征值。[提示:使矩阵 $\begin{pmatrix} ha_0 & g \\ g & ha_0 \end{pmatrix}$ 对角化]

\subsection{7.26}
质量为 $\mu$ 的粒子处于三维各向同性谐振子势 $$V(r) = \frac{1}{2} \mu \omega^2 r^2$$ 中。(1)求粒子的本征能量和相应的简并度;(2)如再加上微扰 $\hat{H}' = bxz$ 的作用 ($b$ 为小的正实数),求粒子的基态和第一激发态能量的一级修正;(3)如在以上势阱(含微扰)中,放入 5 个全同无相互作用的自旋 $s = 0$ 的粒子,求体系基态能量;(4)如在以上势阱(含微扰)中,放入 5 个全同无相互作用的自旋 $s = 1/2$ 的粒子,求体系基态能量。

\subsection{7.27}
$a_i^+, a_j$ 是玻色子在单粒子态 $\phi_i (i = 1, 2)$ 上的产生算符与湮没算符,满足对易关系 $[a_i, a_j] = [a_i^+, a_j^+] = 0,  [a_i, a_j^+] = \delta_{ij}$。令 $$J_x = \frac{h}{2} (a_1^+ a_2 + a_2^+ a_1),  J_y = \frac{i h}{2} (a_1^+ a_2 - a_2^+ a_1),  J_z = \frac{h}{2} (a_2^+ a_2 - a_1^+ a_1)$$ (1) 证明 $$[J_x, J_y] = i h J_z,  [J_y, J_z] = i h J_x,  [J_z, J_x] = i h J_y$$ (2) 证明 $$\hat{J}^2 = \hat{J}_x^2 + \hat{J}_y^2 + \hat{J}_z^2 = \frac{\hat{N}_1 + \hat{N}_2}{2} \left( \frac{\hat{N}_1 + \hat{N}_2}{2} + 1 \right) h^2$$, $$[\hat{J}^2, \hat{J}_\alpha] = 0, \alpha = x,y,z$$,其中 $\hat{N}_1 = a_1^+ a_1$ 与 $\hat{N}_2 = a_2^+ a_2$ 分别是 $\phi_1$ 与 $\phi_2$ 态上粒子占有数算符;(3) 求 $\hat{J}^2$ 与 $\hat{J}_z$ 的共同本征态,以及它们的本征值。