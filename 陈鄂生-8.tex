\section{第八章}

\subsection{8.1}
求低能(s 波)粒子在球方势垒上散射总截面 $\sigma_t$。已知 $V(r) = 
\begin{cases}
V_0, & r < a \\
0, & r > a 
\end{cases}$,
$V_0 > 0$,并给出 $\sigma_t$ 在 $E \to 0$,$V_0 \to \infty$ 时(刚球散射)的极限值。

\subsection{8.2}
求低能(s波)波球方势阱散射总截面 $\sigma_t$,势场为 $V(r) = 
\begin{cases}
-V_0, & r < a \\
0, & r > a
\end{cases}$,其中 $V_0 > 0$,并求 $\sigma_t$ 在 $E \to 0$ 时的极限值。

\subsection{8.3}
质量为 $\mu$ 的粒子被中心力场 $V(r) = \frac{\alpha}{r^2} (\alpha > 0)$ 散射。(1)求各分波的相移 $\delta_l$;(2)在 $\mu \alpha / \hbar^2 \ll 1$ 条件下,求 $\delta_l$ 的渐近式,并计算 $E \to 0$ 时 s 波散射总截面 $\sigma_t$,及任意能量 E 时的散射微分截面 $\sigma(\theta)$。

\subsection{8.4}
求低能 ($ka \ll 1$) 粒子在势场 $V(r) = \pm V_0 \delta(r-a)$ ($V_0 > 0$) 上的 s 波散射总截面 $\sigma_t$。

\subsection{8.5}
用玻恩近似法计算下列势场中的散射微分截面 $\sigma(\theta)$:

(1) $V(r) = 
\begin{cases}
V_0, & r < a \\
0, & r > a
\end{cases}$;
(2) $V(r) = \frac{Be^{-ar}}{r}, \quad a > 0$;
(3) $V(r) = V_0 e^{-ar}, \quad a > 0$;
(4) $V(r) = \frac{A}{r^2}$;
(5) $V(r) = V_0 \delta(r - a), \quad a > 0$.

\subsection{8.6}
试用玻恩近似公式计算库仑散射的微分截面 $\sigma(\theta)$,库仑势为 $V(r)=\alpha/r$,入射粒子质量为 $\mu$,速度为 $v$,$\alpha$ 为实数。

\subsection{8.7}
质量为 $\mu$ 的高能粒子被中心力势 $V(r)=Ae^{-r^2/a^2}(A>0,a>0)$ 散射,求散射微分截面 $\sigma(\theta)$ 与总截面 $\sigma_t$。

\subsection{8.8}
一束中子射向氢分子而发生弹性散射,氢分子中两个原子核同中子的作用可以用下面的简化势代替:$V(r) = -V_{0} \left[ \delta (r-a) + \delta (r+a) \right]$,其中 $V_{0}$ 为正的常数,$a$ 与 $-a$ 分别为两个原子核的位置矢量。求高能中子散射的微分截面 $\sigma(\theta, \varphi)$,并指出截面取极大值的方向。

\subsection{8.9}
质量为 $\mu$ 电荷为 $Q$ 的粒子被一个势场 $V(r)$ 散射。此势场是一个球对称电荷分布 $\rho(r)$ 产生的静电势场。设 $\rho(r)$ 随 $r \to \infty$ 很快趋于零,并有 $\int \rho(r)dr = 0$ 和 $\int r^2 \rho(r)dr = A$ ($A$为已知常数)。试用玻恩近似计算向前散射的微分截面 $\sigma(0)$.

\subsection{8.10}
考虑低能 s 波 n-p 散射,作用力势 $V = V_0(r) + g(r) \hat{S}_1 \cdot \hat{S}_2$,其中 $r$ 是两个粒子之间的距离,$\hat{S}_1$ 与 $\hat{S}_2$ 分别是入射中子与靶质子的自旋。已知 s 波自旋三重态出射球面波振幅为 $f_3$,自旋单态出射球面波振幅为 $f_1$。(1)设入射中子处于自旋极化态 $s_z = h/2$,靶质子处于自旋极化态 $s_z = -h/2$,求散射总截面。(2)设入射中子处于自旋极化态 $s_z = h/2$,靶质子是非极化的,求散射总截面,中子自旋取向不变的散射总截面,中子自旋反向的散射总截面。

\subsection{8.11}
假想一个能量 $E \to 0$ 的中子-中子散射,相互作用力势为
$$V = 
\begin{cases}
V_0 \sigma_1 \cdot \sigma_2, & r < a \\
0, & r > a
\end{cases}$$
其中 $\sigma_1$ 与 $\sigma_2$ 是入射中子与靶中子的泡利矩阵,$V_0$ 是常数。入射中子与靶中子都是非极化的。计算散射总截面 $\sigma_t$。

\subsection{8.12}
考虑两个质量为 $m$ 的高能全同粒子散射,相互作用力势为 $V(r) = Ae^{-\alpha r}/r$,其中 $A$ 与 $\alpha$ 是大于 $0$ 的常数。分别在以下情况,用玻恩近似公式计算散射微分截面 $\sigma(\theta)$: (1) 粒子自旋 $s = 0$;(2) 粒子自旋 $s = 1/2$,并且散射是非极化的;(3) 粒子自旋 $s = 1/2$,并且这两个粒子的自旋均指向 $z$ 轴正方向。(提示:$\int_0^\infty e^{-\alpha x} \sin bx dx = \frac{b}{a^2 + b^2}, a > 0$)

\subsection{8.13}
考虑两个质量为 $m$,自旋 $s = 1/2$ 的高能全同粒子散射。两个粒子之间的相互作用力势为 $V = \frac{Ae^{-\alpha r}}{r} \hat{S}_1 \cdot \hat{S}_2$,其中 $\hat{S}_1$ 与 $\hat{S}_2$ 分别是入射粒子与靶粒子的自旋,$A$ 与 $\alpha$ 是正实数。分别在以下情况用玻恩近似公式计算散射微分截面 $\sigma(\theta)$:
(1) 两个粒子是同向极化的;(2) 两个粒子是反向极化的;(3) 两个粒子是非极化的。(提示:$\int_0^\infty e^{-\alpha r} \sin bx dx = \frac{b}{a^2 + b^2}$, $a > 0$)

\subsection{8.14}
质量为 $m$ 无自旋的粒子受到中心力势 $V(r) = -\frac{\hbar^2}{ma^2} \frac{1}{\cosh^2 (r/a)}$ 的散射,其中 $a$ 是常数。已知方程 
$$\frac{d^2 y(x)}{dx^2} + K^2 y(x) + \frac{2}{\cosh^2 x} y(x) = 0$$ 
的两个线性独立解为 $y(x) = e^{\pm iKx} (\tanh x \mp iK)$。在低能下,求粒子能量为 $E$ 时,$s$ 分波的散射截面及其角分布。