\section{表象}

\subsection{3.1}
在 $p$ 表象求解 $\delta$ 势阱 $V(x) = -\gamma \delta(x)$ 中的定态能量和波函数 $(\gamma > 0)$。

\subsection{3.2}
已知在 $\delta$ 势阱 $V(x) = - \gamma \delta (x)$ 中的定态归一化波函数 ($p$ 表象) 为
$$\varphi(p) = \frac{A}{p^2 + \hbar^2 k^2}, \quad A = \sqrt{\frac{2 \hbar^3 k^3}{\pi}}, \quad k = \frac{\mu \gamma}{\hbar^2}$$
试计算 $\Delta x \Delta p$,验证测不准关系。

\subsection{3.3}
在 $p$ 表象计算一维谐振子定态能量和波函数。

\subsection{3.4}
质量为 $\mu$ 的粒子在均匀力场 $f(x) = -F(F > 0)$ 中运动,运动的范围限制在 $x > 0$。给出动量表象中的定态方程,并求出定态波函数 $\varphi(p)$。

\subsection{3.5}
质量为 $\mu$ 的粒子在均匀力场 $f(x) = -F(F > 0)$ 中运动,$\rho(p, t)$ 为其在动量空间的概率密度,求 $\partial \rho / \partial t $ 与 $\partial \rho / \partial p$ 的关系。

\subsection{3.6}
已知 $t = 0$ 时一维自由粒子波函数在坐标表象和动量表象分别是
$$\psi(x) = Nx \exp \left( -ax^2 + i \frac{p_0 x}{\hbar} \right)$$
$$\varphi(p) = C(p - p_0) \exp \left[ -b(p - p_0)^2 \right]$$
其中 $a, b$ 和 $p_0$ 都是已知实数,$N$ 与 $C$ 是归一化常数。试求 $t = 0$ 时和 $t > 0$ 时粒子坐标和动量的平均值 $\overline{x}$ 和 $\overline{p}$。

\subsection{3.7}
中子 $n$ 和反中子 $\bar{n}$ 的质量都是 $m$,它们的态 $|n\rangle$ 和 $|\bar{n}\rangle$ 可以看成是一自由哈密顿量 $H_0$ 的简并态:$H_0 |n\rangle = mc^2 |n\rangle, \quad H_0 |\bar{n}\rangle = mc^2 |\bar{n}\rangle$。设有某种相互作用 $\hat{H}'$ 能使中子与反中子相互转变:$\hat{H}' |n\rangle = \alpha |\bar{n}\rangle, \quad \hat{H}' |\bar{n}\rangle = \alpha^{*} |n\rangle$,其中 $\alpha = \alpha^{*}$。试求 $t = 0$ 时刻的一个中子在 $t$ 时刻变成反中子的概率。

\subsection{3.8}
有一量子体系,其态矢空间三维。选择基矢 $\{|1\rangle, |2\rangle, |3\rangle\}$。体系的哈密顿量 $\hat{H}$ 及另外两个力学量 $\hat{A}$ 与 $\hat{B}$ 为
$$\hat{H} = \hbar \omega_0 
\begin{pmatrix}
1 & 0 & 0 \\
0 & 2 & 0 \\
0 & 0 & 2
\end{pmatrix}, \quad \hat{A} = a 
\begin{pmatrix}
1 & 0 & 0 \\
0 & 0 & 1 \\
0 & 1 & 0
\end{pmatrix}, \quad \hat{B} = b 
\begin{pmatrix}
0 & 1 & 0 \\
1 & 0 & 0 \\
0 & 0 & 1
\end{pmatrix}$$
设 $t = 0$ 时体系的态矢为 $|\psi(0)\rangle = \frac{1}{\sqrt{2}} |1\rangle + \frac{1}{2} |2\rangle + \frac{1}{2} |3\rangle$,(1)在 $t = 0$ 时测量体系能量 $H$ 可得哪些结果?相应概率多大?计算 $H$ 平均值 $\overline{H}$ 及 $\Delta H = \sqrt{\overline{H^2 - (\overline{H})^2}}$。(2)如 $t = 0$ 时测量 $A$,可能值与相应概率有多大?写出测量后体系的态矢量。(3)计算任意 $t$ 时刻 $A$ 与 $B$ 的平均值 $A(t)$ 与 $B(t)$。

\subsection{3.9}
厄米算符 $\hat{A}$ 与 $\hat{B}$ 满足 $\hat{A}^2 = \hat{B}^2 = 1$,且 $\hat{A} \hat{B} + \hat{B} \hat{A} = 0$。求 (1) 在 $\hat{A}$ 表象中算符 $\hat{A}$ 与 $\hat{B}$ 的矩阵表示;(2) 在 $\hat{A}$ 表象中算符 $\hat{B}$ 的本征值与本征态矢;(3) 由 $\hat{A}$ 表象到 $\hat{B}$ 表象的幺正变换 S 矩阵,并把 $\hat{B}$ 矩阵对角化。

\subsection{3.10}
在 $l = 1$ 的 $(\hat{L}_x, \hat{L}_y)$ 表象中, 基矢为
$$\varphi_1 = Y_{11}(\theta, \varphi), \quad \varphi_2 = Y_{10}(\theta, \varphi), \quad \varphi_3 = Y_{1-1}(\theta, \varphi)$$
求 $\hat{L}_x, \hat{L}_y$ 与 $\hat{L}_z$ 的矩阵表示。

\subsection{3.11}
已知在 $l=1$ 的 $(\hat{L}^2, \hat{L}_z)$ 表象中,
$$\hat{L}_x = \frac{\hbar}{\sqrt{2}} \begin{pmatrix}
0 & 1 & 0 \\
1 & 0 & 1 \\
0 & 1 & 0
\end{pmatrix}, \quad \hat{L}_y = \frac{\hbar}{\sqrt{2}} \begin{pmatrix}
0 & -i & 0 \\
i & 0 & -i \\
0 & i & 0
\end{pmatrix}, \quad \hat{L}_z = \hbar \begin{pmatrix}
1 & 0 & 0 \\
0 & 0 & 0 \\
0 & 0 & -1
\end{pmatrix}$$
(1) 给出它们的本征值与本征态矢;(2) 给出 $(\hat{L}^2, \hat{L}_z)$ 表象到 $(\hat{L}^2, \hat{L}_x)$ 表象变换的 S 矩阵,并通过 S 矩阵,求出在 $(\hat{L}^2, \hat{L}_x)$ 表象中 $\hat{L}_x, \hat{L}_y$ 与 $\hat{L}_z$ 的矩阵表示,本征值与本征态矢。

\subsection{3.12}
有一量子体系处于角动量 $\hat{L}^2$ 与 $\hat{L}_z$ 的共同本征态上,总角动量平方值为 $2\hbar^2$。已知测量 $\hat{L}_y$ 得值为 0 的概率是 1/2, 求测量 $\hat{L}_y$ 得值为 $\hbar$ 的概率。

\subsection{3.13}
粒子处于态 $\psi = Ce^{-\alpha r}(x+y+2z)$,其中 $\alpha$ 为正数,$C$ 为归一化常数。求 $L^2$ 的取值,$L_z$ 的平均值,$L_z = \hbar$ 的概率,$L_x$ 的可能值及相应概率。

\subsection{3.14}
体系处于态 $\psi = c_1 Y_{11} + c_2 Y_{10} \ (|c_1|^2 + |c_2|^2 = 1)$, 求 (1) $L_z$ 的可能值及相应概率;(2) $L^2$ 的可能值及相应概率;(3) $L_x$ 的可能值及相应概率。

\subsection{3.15}
体系处于态 $\psi = c_1 Y_{11} + c_2 Y_{20} $ ($|c_1|^2 + |c_2|^2 = 1$),求 (1) $L_z$ 的可能值及相应概率;(2) $L^2$ 的可能值及相应概率;(3) $L_x$ 的可能值及相应概率。

\subsection{3.16}
在角动量 $\hat{J}^2$ 与 $\hat{J}_z$ 为对角矩阵的表象,对 $j = 3/2$,求 $\hat{J}^2, \hat{J}_x, \hat{J}_y$ 和 $\hat{J}_z$ 的矩阵表示。

\subsection{3.17}
在 $ j=3/2 $ 的 $(\hat{J}^2 \hat{J}_z)$ 表象中,$\hat{J}_x$ 的矩阵为
$$\hat{J}_x = \frac{\hbar}{2} \begin{pmatrix}
0 & \sqrt{3} & 0 & 0 \\
\sqrt{3} & 0 & 2 & 0 \\
0 & 2 & 0 & \sqrt{3} \\
0 & 0 & \sqrt{3} & 0
\end{pmatrix}$$
其中行和列都是按 $\hat{J}_z$ 的量子数 $m$ 由大到小排列的。(1)求出 $\hat{J}_y$ 的矩阵;(2)求出与 $\hat{J}_y$ 最大本征值相应的本征态,并说明其中各矩阵元的物理意义。

\subsection{3.18}
在由正交基矢 $\{|1\rangle, |2\rangle, |3\rangle\}$ 构成的三维态矢空间中,哈密顿算符 $\hat{H}$ 与力学量 $\hat{A}$ 的矩阵为
$$\hat{H} = E_0 
\begin{pmatrix}
1 & 0 & 0 \\
0 & -1 & 0 \\
0 & 0 & -1
\end{pmatrix}, \quad \hat{A} = a 
\begin{pmatrix}
1 & 0 & 0 \\
0 & 0 & 1 \\
0 & 1 & 0
\end{pmatrix}$$
(1)证明 $\hat{A}$ 为守恒量;(2)求出 $\hat{H}$ 与 $\hat{A}$ 的共同本征态矢组。

\subsection{3.19}
一个空间转子,其哈密顿量为 $\hat{H} = \frac{\hat{L}_x^2}{2I_x} + \frac{\hat{L}_y^2}{2I_y} + \frac{\hat{L}_z^2}{2I_z}$。转子的轨道角动量量子数 $l=1$, $I_x$, $I_y$ 与 $I_z$ 均为正实数。(1)在角动量表象中求出 $\hat{L}_x, \hat{L}_y$ 与 $\hat{L}_z$ 的矩阵表示;(2)求出 $\hat{H}$ 的本征值。

\subsection{3.20}
质量为 $\mu$ 的粒子受到力 $F(r) = -\nabla V(r)$ 的作用,粒子的波函数满足动量空间薛定谔方程
$$i\hbar \frac{\partial}{\partial t} \phi(p, t) = \left( \frac{p^2}{2\mu} - \alpha \nabla_p^2 \right) \phi(p, t)$$
其中 $\alpha$ 是实常数。求 $F(r)$。