\section{第五章}

\subsection{5.1}
带有电荷 $q$ 的一维谐振子,其哈密顿量为
$$\hat{H} = -\frac{\hbar^{2}}{2\mu} \frac{d^{2}}{dx^{2}} + \frac{1}{2} \mu \omega^{2} x^{2}$$
振子受到恒定弱电场 $\mathbf{E} = \epsilon \mathbf{i}$ 的作用。电场的作用可视为微扰。计算二级近似能量和一级近似波函数。如果振子是各向同性介质中的离子(带有电荷 $q$),计算由电场 $\epsilon$ 引起的极化率。

\subsection{5.2}
已知体系的哈密顿量在某力学量表象中表示为
$$\hat{H} = E_0 \begin{pmatrix} 0 & 1 & 0 \\ 1 & 0 & 1 \\ 0 & 1 & 0 \end{pmatrix} + \epsilon \begin{pmatrix} 0 & 0 & 0 \\ 0 & 0 & 1 \\ 0 & 1 & 0 \end{pmatrix}$$
其中 $E_0, \epsilon > 0, \epsilon \ll E_0$。试用微扰方法求二级近似能量和一级近似态矢。

\subsection{5.3}
在 $\hat{H}_0$ 表象中 $\hat{H}$ 的矩阵为
$$\hat{H} = \begin{pmatrix} E_1^{(0)} & 0 & a \\ 0 & E_2^{(0)} & b \\ a^* & b^* & E_3^{(0)} \end{pmatrix}$$
其中 $E_i^{(0)}(i=1,2,3)$ 为实数,且 $|a|, |b|$ 比 $|E_i^{(0)}|$ 小得多。试用微扰论求能量至二级近似。

\subsection{5.4}
考虑到类氢原子核不是点电荷,而是半径为 $R$ 的均匀带电 $Ze$ 的球体,用微扰方法计算这种效应对类氢原子基态能量的一级修正。已知电子在球形核电场的势能为
$$V(r) = \begin{cases} \frac{Ze^2}{r}, & r > R \\ \frac{Ze^2}{2R} \left( \frac{r^2}{R^2} - 3 \right), & r < R \end{cases}$$
如果类氢原子核是半径为 $R$ 的均匀带电球面,结果又如何?

\subsection{5.5}
一维无限深方势阱($0 < x < a$)中的粒子受到微扰 $\hat{H}' = A \cos(\pi x / a)$ ($0 < x < a$) 的作用,其中 $A$ 为实常数,求基态能量的二级近似与波函数的一级近似。

\subsection{5.6}
一维谐振子 $\left( V = \frac{1}{2} \mu \omega^{2} x^{2} \right)$ 受到微扰 $\hat{H}' = \frac{1}{2} \lambda \mu \omega^{2} x^{2}$ 的作用 ($|\lambda| \ll 1$),计算能量的三级近似值,并与严格值比较。

\subsection{5.7}
试求哈密顿量为 $\hat{H} = -\frac{\hbar^2}{2\mu} \frac{d^2}{dx^2} + \frac{1}{2} \mu \omega^2 x^2 + ax^3 + bx^4$ 的体系的一级近似能量,其中 $a$ 与 $b$ 是小的常数。

\subsection{5.8}
粒子的哈密顿量 $\hat{H} = \hat{H}_0 + \hat{H}'$。$\hat{H}_0$ 与 $\hat{H}'$ 在 $Q$ 表象中的矩阵为
$$\hat{H}_0 = E_0 \begin{pmatrix} 2 & 0 & 1 \\ 0 & 2 & 0 \\ 1 & 0 & 2 \end{pmatrix}, \quad \hat{H}' = \varepsilon \begin{pmatrix} 0 & 0 & 0 \\ 0 & 2 & 0 \\ 0 & 0 & 2 \end{pmatrix}$$
其中 $E_0$ 为正实数,$|\varepsilon| \ll E_0$,$\hat{H}'$ 为微扰。(1)忽略微扰,求出 $\hat{H}_0$ 的本征值与本征态矢;(2)考虑微扰,求出基态的二级近似能量和一级近似态矢。

\subsection{5.9}
一个质量为 $\mu$ 的粒子在一维势场中运动,势函数为
$$V(x) = \begin{cases} \infty, & |x| > 3a \\ 0, & a < x < 3a \\ V_0, & |x| < a \\ 0, & -3a < x < -a \end{cases}$$
将 $V_0$ 部分视为在宽度为 $6a$ 的无限深方势阱 ($V = 0, |x| < 3a; V = \infty, |x| > 3a$) 中的微扰,用微扰方法求基态一级近似能量。

\subsection{5.10}
体系的哈密顿量 $\hat{H} = \hat{H}_0 + \hat{H}'$,其中 $\hat{H}' = i \lambda [\hat{A}, \hat{H}_0]$ 是微扰,$\hat{A}$ 是厄米算符,$\lambda$是实数。令$\hat{B}$是另一个厄米算符,$\hat{C} = i[\hat{B}, \hat{A}]$。(1)已知在 $\hat{H}_0$ 的非简并的基态上,算符 $\hat{A}, \hat{B}$ 与 $\hat{C}$ 的平均值为 $\langle A \rangle_0, \langle B \rangle_0, \langle C \rangle_0$。考虑微扰后,计算$\hat{B}$在基态上的平均值 $\langle B \rangle$ 到$\lambda$的一级近似。(2)在以下问题中检验你的结果,
$$\hat{H}_0 = \sum_{k=1}^3 \left( \frac{\hat{p}_k^2}{2\mu} + \frac{1}{2}\mu \omega^2 x_k^2 \right), \quad \hat{H}' = \lambda x_3$$
计算 $\langle x_k \rangle$ (基态)到$\lambda$的一级近似,用 $\langle x_k \rangle$ 的严格值同你的结果比较。

\subsection{5.11}
一维谐振子哈密顿量 $\hat{H} = -\frac{\hbar^2}{2\mu}\frac{d^2}{dx^2} + \frac{1}{2}\mu\omega^2x^2$。微扰 $\hat{H}' = \frac{\lambda}{2}\mu\omega^2x^{2n}$ ($n$为正整数,$\lambda$为微小量)。(1)求出基态能量的一级和二级修正值;(2)将$n=1$的结果同$\hat{H} = \hat{H}_0 + \hat{H}'$的精确结果比较。

\subsection{5.12}
可以证明若点电荷在静电场中的势能为 $V(r)$,则均匀带电小球在静电场中的势能为 $V(r) + \frac{1}{6} r_0^{2} \nabla^2 V(r)$,其中 $r_0$ 是小球的半径,$r$ 是球心的位置。试利用这一结果,计算氢原子1s态能级由于不是点电荷而带来的修正。已知 $\psi_{1s}(r) = (\pi a^3)^{-1/2} e^{-r/a}$,玻尔半径 $a = 0.529 \, \text{\AA}$。取 $r_0 = e^2 / m_e c^2$ (电子的经典半径),其中 $e^2 = 1.44 \, \text{MeV} \cdot \text{fm}$,$m_e = 0.511 \, \text{MeV}/c^2$。

\subsection{5.13}
质量为 $\mu$ 的粒子在一维谐振子势场 $V(x) = \frac{1}{2} \mu \omega^2 x^2$ 中运动。在动能
$$T = \frac{p^2}{2\mu}$$
的非相对论近似下,定态能量为 $E_n = \left( n + \frac{1}{2} \right) \hbar \omega$,定态波函数为
$\psi_n (x) = N_n e^{-\alpha^2 x^2 / 2} H_n (\alpha x)$,其中 $\alpha = \sqrt{\mu \omega / \hbar}$,$N_n$ 为归一化常数。考虑 $T$ 与 $p$ 的相对论修正, 计算能级 $E_n$ 的移动 $\Delta E$ 至 $1/c^2$ 阶。(提示:
$$\frac{d^2 \psi_n}{dx^2} = \frac{\alpha^2}{2} \left[ \sqrt{n(n-1)} \psi_{n-2} - (2n+1)\psi_n + \sqrt{(n+1)(n+2)}\psi_{n+2} \right]$$

\subsection{5.14}
电子在类氢离子势场 $V(r) = -\frac{Ze^2}{r}$ 的定态能量为 $E_n = -\frac{Z^2 e^2}{2an^2}$,定态波函数为 $\psi_{nlm}(r)$。这是在动能 $T = p^2 / 2\mu$ 的非相对论近似下得到的结果。现考虑 $T$ 与 $p$ 的相对论修正,计算能级 $E_n$ 的移动 $\Delta E$ 至 $1/c^2$ 阶。
[提示:$\left\langle \frac{1}{r} \right\rangle_{nlm} = \frac{Z}{an^2}$, $\left\langle \frac{1}{r^2} \right\rangle_{nlm} = \frac{2Z^2}{(2l+1)a^2 n^3}$]

\subsection{5.15}
一原子在z向磁场$B$中除了能级的塞曼分裂外,还受到微扰 $\Delta \hat{H}_d = \frac{\mu_B^2 B^2}{2e^2 a} r^2 \sin^2 \theta$ 的作用,其中 $\mu_B$ 为玻尔磁子,$a$ 为玻尔半径,$e$ 为电子电荷。(1)已知氢原子基态波函数 $\psi_{100} = (\pi a^3)^{-1/2} e^{-r/a}$,求一级微扰能 $\Delta E_d$;(2)估计这项修正的量级(设 $B = 10^4$ 高斯),同塞曼分裂($\mu_B B$ 量级)比较;(3)分析这项修正的物理意义。

\subsection{5.16}
转动惯量为$I$电偶极矩为$D$的平面转子绕z轴转动,体系的哈密顿量 $\hat{H}_0 = \frac{\hat{L}_z^2}{2I}$,定态能量 $E_m^{(0)} = \frac{m^2 \hbar^2}{2I}$,定态波函数 $\psi_m^{(0)} = \frac{1}{\sqrt{2\pi}} e^{im\varphi}$,$m = 0, \pm 1, \cdots$。如果在x方向存在均匀弱电场 $\mathbf{E} = \epsilon \mathbf{i}$,电偶极矩同电场的作用 $\hat{H}' = -D \cdot \mathbf{E} = -D \epsilon \cos \varphi$ 可视为微扰,计算二级近似能量和一级近似波函数。

\subsection{5.17}
一根质量均匀分布长度为$d$的杆,以它的中心为固定点,被约束在一平面上转动。此杆具有质量$M$和固定于两端点的电荷$Q$与$-Q$。(1)给出此体系的哈密顿量及其本征函数与本征值;(2)如有一个处于该转动平面的恒定弱电场$\epsilon$作用于这个体系,用微扰方法求基态新的本征函数(一级近似)与本征能量(二级近似);(3)如果外电场很强,求基态近似波函数与能量。

\subsection{5.18}
转动惯量为$I$电偶极矩为$D$的空间转子绕固定点$O$转动,体系的哈密顿量为 $\hat{H}_0 = \frac{\hat{L}^2}{2I}$,定态能量为 $E_l^{(0)} = \frac{l(l+1)\hbar^2}{2I}$,定态波函数为 $Y_{lm}(\theta, \varphi)$。如果在z方向存在均匀弱电场 $\mathbf{E} = \epsilon \mathbf{k}$,电偶极矩同电场的作用 $\hat{H}' = -D \cdot \mathbf{E} = -D \epsilon \cos \theta$ 可视为微扰,计算能量的二级近似值。

\subsection{5.19}
空间转子作受偶转动,哈密顿量为 $\hat{H} = A\hat{L}^2 + B\hbar^2 \cos 2\varphi$,其中 $A$ 与 $B$ 为正实数,且 $A \gg B$。试计算 $p$ 能级 ($l=1$) 的分裂,及零级近似波函数。

\subsection{5.20}
设在表象 $\hat{H}_0$ 中,$\hat{H}_0$ 与微扰 $\hat{H}'$ 的矩阵为
$$\hat{H}_0 = E_0 \begin{pmatrix} 1 & 0 & 0 \\ 0 & 1 & 0 \\ 0 & 0 & 2 \end{pmatrix}, \quad \hat{H}' = \varepsilon \begin{pmatrix} 2 & 1 & 3 \\ 1 & 2 & 3 \\ 3 & 3 & 1 \end{pmatrix}$$
其中 $E_0$ 与 $2E_0$ 是基态与激发态的零级近似能量,$\varepsilon$ 是微小量。(1)求基态的一级近似能量与零级近似态矢;(2)求激发态的二级近似能量与一级近似态矢。

\subsection{5.21}
质量为 $\mu$ 的粒子在 $xy$ 面上运动,其哈密顿量为
$$\hat{H} = \frac{1}{2\mu} (\hat{p}_x^2 + \hat{p}_y^2) + \frac{1}{2}\mu\omega^2 (x^2 + y^2) + \lambda xy$$
其中 $\lambda$ 是小的实数,$\lambda xy$ 可视为微扰。试计算能量为 $2\hbar\omega$ 的能级的分裂。

\subsection{5.22}
处于三维各向同性谐振子第一激发态的粒子, 受到微扰 $\hat{H}' = \lambda xy$ 的作用, 其中 $\lambda$ 为常数, 求能量的一级修正.

\subsection{5.23}
设硼原子(原子序数为5)受到 $\hat{H}' = f(r)xy$ 的微扰作用,在一级近似下,(1)问价电子2p能级分裂成几个能级?(2)如已知其中一个能级的移动值 $A > 0$,求其余各能级的移动值;(3)求出各能级对应的波函数,用原来的2p态波函数 $\psi_{211}, \psi_{210}$ 与 $\psi_{21-1}$ 表示。

\subsection{5.24}
边长为 $a$ 的刚性立方势中电子具有能量 $3\pi^2 \hbar^2 / \mu a^2$。如微扰哈密顿量 $\hat{H}' = bxy$,试求它对能量的一级修正,$b$为常数。

\subsection{5.25}
一粒子在二维无限深方势阱中运动,$V(x, y) = \begin{cases} 0, & 0 < x, y < a \\ \infty, & \text{其他区} \end{cases}$。设加上微扰 $\hat{H}' = \lambda xy (0 < x, y < a)$,求基态和第一激发态的一阶能量修正。

\subsection{5.26}
(1)设氢原子处于沿 z 方向的均匀静磁场 $\mathbf{B} = B \mathbf{k}$ 中。不考虑自旋,在弱磁场下,求 $n = 2$ 能级的分裂情况。(2)如果沿 z 方向不仅有静磁场 $\mathbf{B} = B \mathbf{k}$,还有均匀静电场 $\mathbf{E} = \epsilon \mathbf{k}$,用微扰方法求能级的分裂情况(一级近似)。(3)如果电场方向沿 z 轴,$\mathbf{E} = \epsilon \mathbf{k}$,磁场方向沿 x 轴,$\mathbf{B} = B \mathbf{i}$,再求能级的分裂情况(一级近似)。假若将电场与磁场方向对换,结果又如何?(提示:$\langle 200|z|210\rangle = -3a$)

\subsection{5.27}
考虑一个二维谐振子,哈密顿量为 $\hat{H} = \frac{1}{2} (\hat{p}_x^2 + \hat{p}_y^2) + \frac{1}{2} (x^2 + y^2)$。已知其最低 3 个能量的本征态为
$$\psi_{00} = \sqrt{\frac{1}{\pi}} e^{-(x^2 + y^2)/2}, \psi_{10} = \sqrt{\frac{2}{\pi}} x e^{-(x^2 + y^2)/2}, \psi_{01} = \sqrt{\frac{2}{\pi}} y e^{-(x^2 + y^2)/2}$$
设有一微扰 $V(x, y) = \frac{1}{2} \epsilon x y (x^2 + y^2)$ ($\epsilon \ll 1$),试对上述态计算出 $V$ 引起的一级微扰修正。

\subsection{5.28}
二维谐振子体系哈密顿量 $\hat{H}_0 = \frac{1}{2\mu} (\hat{p}^2_x + \hat{p}^2_y) + \frac{1}{2}\mu\omega^2 (x^2 + y^2)$。体系还受到微扰 $\hat{H}' = \epsilon \mu \omega^2 x y$ 的作用,其中 $\epsilon$ 是正的实数,$\epsilon \ll 1$。求体系基态能量(二级近似)和波函数(一级近似)。再求精确能量和波函数,比较近似能量和精确能量.

\subsection{5.29}
在谐振子的哈密顿量 $ \hat{H}_0 = \frac{1}{2\mu} \hat{p}^2 + \frac{1}{2} \mu \omega^2 x^2 $ 上,加上微扰项 $ \hat{H}' = \lambda x^3 $,求能量的二级修正。

\subsection{5.30}
一根长为$l$无质量的绳子一端固定,另一端系质量为$m$的质点。在重力作用下,质点在竖直平面内摆动。(1)写出质点运动的哈密顿量;(2)在小角度下求系统的能级;(3)求由于小角度近似的误差而产生的基态能量最低阶修正。

\subsection{5.31}
一个对称陀螺,转动惯量 $I_x = I_y, I_z \neq I_x$。(1)只考虑转动,利用角动量算符写出体系的哈密顿量算符,求出本征态、本征能量和简并度;(2)考虑不对称陀螺,$I_x \neq I_y \neq I_z, I_x = I + (a/2), I_y = I - (a/2)$,其中 $|a| \ll I$。对于 $l = 1$ 的本征态,计算由 $a \neq 0$ 引起的对能量的修正,要求精确到 $a$ 的一阶。

\subsection{5.32}
一维线性谐振子能量算符 $\hat{H}_0 = -\frac{\hbar^2}{2m}\frac{d^2}{dx^2} + \frac{1}{2} m\omega^2 x^2$,本征方程
$\hat{H}_0 \phi_n = E_n^{(0)} \phi_n$。现加一微扰 $\hat{H}' = \lambda \omega(x\hat{p} + \hat{p}x)/2$,其中 $\lambda \ll 1$。求体系能量 $E_k$ 至二级近似,基态与第一激发态波函数至一级近似。

\subsection{5.33}
粒子在中心力场 $V(r)$ 中运动,本征方程为 $\hat{H}_0 | nlm \rangle = E^{(0)}_n | nlm \rangle$。若在 $\hat{H}_0$ 上依次加上 $\hat{H}_1 = \alpha (\hat{L}^2_x + \hat{L}^2_y)$ 与 $\hat{H}_2 = \beta \hat{L}^2_y$ ($\alpha, \beta$均为正实数,且 $\beta \ll \alpha$)。(1)求 $\hat{H} = \hat{H}_0 + \hat{H}_1$ 的本征值与本征函数,能量简并度;(2)对 $n = 3,l = 1$,求 $\hat{H} = \hat{H}_0 + \hat{H}_1 + \hat{H}_2$ 的本征值至一级近似,并求零级近似波函数。

\subsection{5.34}
粒子的哈密顿量 $\hat{H} = \hat{H}_0 + \hat{H}'$, $\hat{H}_0 = a (\hat{L}^2 + \hbar \hat{L}_z)$, $\hat{H}' = b \hbar \hat{L}_x$, 其中 $a, b > 0, b \ll a$。(1)不考虑微扰 $\hat{H}'$, 给出 $\hat{H}_0$ 的本征能量和本征函数;(2)考虑微扰 $\hat{H}'$, 计算能量的二级近似和波函数的一级近似。

\subsection{5.35}
某体系能量算符为 $\hat{H}_0$,有两个能级,$E_1^{(0)}$ 二重简并,$E_2^{(0)}$ 无简并,受微扰 $\hat{H}'$ 作用后,能量算符 ($\hat{H}_0$ 表象) 变成
$$\hat{H} = \hat{H}_0 + \hat{H}' = \begin{pmatrix} E_1^{(0)} & 0 & a \\ 0 & E_1^{(0)} & b \\ a & b & E_2^{(0)} \end{pmatrix} (a, b \text{为实数})$$
(1) 用微扰论公式求能级(二级近似); (2) 求能级的精确值,并和微扰论结果比较。

\subsection{5.36}
距离表面 $x$ 处的一个电子受到势场 $V(x) = \begin{cases} -\frac{a}{x}, & x > 0 \\ \infty, & x < 0 \end{cases}$ 的作用,其中 $a$ 是正的常数。略去自旋影响,(1)求基态能量和波函数;(2)现加入一个电场强度为 $\epsilon$ 的弱电场,用微扰论求基态能量的一级修正。

\subsection{5.37}
设 $\hat{H} = \hat{H}_0 + \hat{H}'$, $\hat{H}_0 = \begin{pmatrix} E_1^{(0)} & 0 \\ 0 & E_2^{(0)} \end{pmatrix}$, $\hat{H}' = \begin{pmatrix} a & b \\ b & a \end{pmatrix}$, $a,b$ 为实数。(1)用微扰方法求近似能量(准至三级近似); (2)求严格解,并同近似解比较。

\subsection{5.38}
在 $\hat{H}_0$ 表象中,$\hat{H} = \hat{H}_0 + \hat{H}'$ 的矩阵表示为
$$\hat{H} = \begin{pmatrix} E_1^{(0)} + \varepsilon_1 & a & b \\ a & E_2^{(0)} + \varepsilon_2 & c \\ b & c & E_3^{(0)} + \varepsilon_3 \end{pmatrix}$$
其中 $E_1^{(0)}, E_2^{(0)}, E_3^{(0)}$ 是 $\hat{H}_0$ 的本征值,$\varepsilon_1, \varepsilon_2, \varepsilon_3$ 是微小量,且 $E_1^{(0)} < E_2^{(0)} < E_3^{(0)}$,$a, b, c$ 为实数。试用微扰论求体系的能量至三级修正。

\subsection{5.39}
用变分法计算一维谐振子的基态能量与波函数,谐振子的哈密顿量为 $\hat{H} = -\frac{\hbar^2}{2\mu} \frac{d^2}{dx^2} + \frac{1}{2}\mu\omega^2 x^2$。试按波函数取 $\psi(\lambda, x) = Ne^{-\lambda x^2}$,其中 $\lambda$ 为待定参数,$N$ 为归一化常数。

\subsection{5.40}
质量为 $\mu$ 的粒子在一维势场 $V(z) = \begin{cases} \infty, & z < 0 \\ Gz, & z > 0 \end{cases}$,式中 $G > 0$。(1) 用变分法计算基态能量,在 $z > 0$ 区域内的试探波函数应取下列波函数中的哪一个?为什么?
(a) $z + \lambda z^2$, (b) $e^{-\lambda z^2}$, (c) $ze^{-\lambda z}$, (d) $\sin \lambda z$
(2) 算出基态能量。

\subsection{5.41}
一体系的哈密顿量 $\hat{H} = \hat{T} + V(x), \quad V(x) = \begin{cases} \infty, & x < 0 \\ Ax, & x > 0 \end{cases}, \quad A > 0$。(1) 用变分法取试探波函数 $\psi_1(x) = \sqrt{\frac{2}{b\sqrt{\pi}}} e^{-x^2 / 2b^2}$,求基态能量上限 $E_1$。(2) 已知如果取试探波函数 $\psi_2(x) = \sqrt{\frac{1}{b\sqrt{\pi}}} \frac{2x}{b} e^{-x^2 / 2b^2}$,则基态能量上限为 $E_2 = \left( \frac{81}{4\pi} \right)^{1/3} \left( \frac{A^2 \hbar^2}{\mu} \right)^{1/3}$。对这两个上限,你能接受哪一个?为什么?

\subsection{5.42}
质量为 $\mu$ 的粒子在势场 $V(x) = \begin{cases} \infty, & x < 0 \\ Cx^2, & x > 0 \end{cases}$,$C > 0$ 中运动。(1)用变分法估算粒子基态能量,试探波函数取 $\psi(x) = Axe^{-\lambda x}$,$\lambda$ 为变分参量。(2)它是精确解的上限,还是下限?将它同精确解比较。

\subsection{5.43}
对非简谐振子 $\hat{H} = -\frac{\hbar^2}{2m}\frac{d^2}{dx^2} + \lambda x^4$,取试探波函数 $e^{-\alpha^2 x^2}$,试用变分法求基态能量。

\subsection{5.44}
质量为 $\mu$ 的粒子在势场 $V(x) = -\alpha \delta (x) (\alpha > 0)$ 中运动,以谐振子基态型波函数为试探波函数,求基态近似能量。

\subsection{5.45}
设在氘核中,质子和中子之间的相互作用势为 $V(r) = -V_0 e^{-r/a}$ ($r, a$ 为正实数)。取试探波函数 $\psi = e^{-\lambda r / 2a}$,其中 $\lambda$ 为待定参数。(1)用变分法求基态能量近似值;(2)若 $V_0 = 32.7 \, \text{MeV}, a = 2.16 \, \text{fm}$。试确定 $\lambda$ 的值。

\subsection{5.46}
设 $t < 0$ 时,一维量子体系处于 $\hat{H}_0$ 的某一本征态 $\psi_k$ 上,$t > 0$ 时受到一微弱的外界作用 $\hat{H}'(x,t)$。(1)求 $t > 0$ 时该体系由 $\psi_k$ 态跃迁到 $\hat{H}_0$ 的另一本征态 $\psi_{l}(l \neq k)$的跃迁概率$W_{k \to l}$的一级近似表示式;(2)若$\psi_{k}$为该体系的基态$\psi_{0}$,而 $\hat{H}'(x,t)=F(x)e^{-t/\tau}$,求在$t \gg \tau$时,体系处于某一激发态$\psi_{n}$的概率$W_{0 \to n}$。

\subsection{5.47}
求在方向一致空间均匀,但随时间衰减的电场$\varepsilon(t)=\begin{cases} 0, & t<0 \\ \varepsilon_{0}e^{-t/\tau}, & t>0 \end{cases}$ ($\varepsilon_{0}$与$\tau$为常数,$\tau > 0$)中,原处于基态的氢原子后处于2p态的概率。已知
$$\psi_{100}=\sqrt{\frac{1}{\pi a^{3}}}e^{-r/a}, \quad \psi_{210}=\frac{1}{4a^{2}}\sqrt{\frac{1}{2\pi a}}re^{-r/2a}\cos\theta$$
$$\psi_{211}=-\frac{1}{8a^{2}}\sqrt{\frac{1}{\pi a}}re^{-r/2a}\sin\theta e^{\pm i\varphi}$$

\subsection{5.48}
带有电荷$q$的一维谐振子在光照下发生跃迁。(1)给出电偶极跃迁的选择定则;(2)设照射光的强度为$I(\omega)$,计算振子由基态到第一激发态的跃迁速率.

\subsection{5.49}
计算氢原子在强度为 $I(\omega)$ 的光照下,由基态到 $2p$ 态跃迁的速率。已知氢原子波函数 $\psi_{nlm} = R_{nl}(r)Y_{lm}(\theta,\varphi)$,其中
$$R_{10} = \frac{2}{a^{3/2}}e^{-r/a}, R_{21} = \frac{1}{2\sqrt{6a^{3/2}}}\left(\frac{r}{a}\right)e^{-r/2a}$$
$$Y_{00} = \sqrt{\frac{1}{4\pi}}, Y_{10} = \sqrt{\frac{3}{4\pi}}\cos\theta, Y_{1\pm1} = \pm \sqrt{\frac{3}{8\pi}}\sin\theta e^{\pm i\varphi}$$

\subsection{5.50}
计算氢原子由 2p 态到 1s 态的自发跃迁速率。

\subsection{5.51}
原子两个态之间的电偶极跃迁概率同两个态之间的电偶极跃迁矩阵元绝对值平方成正比。已知氢原子的波函数 $\psi_{nlm}(r) = R_n(r) Y_{lm}(\theta, \varphi)$。取玻尔半径 $a = \hbar^2 / \mu e^2$ 为长度单位,则
$$R_{10} = 2e^{-r}, \quad R_{20} = \left( \frac{1}{2} \right)^{3/2} (2 - r)e^{-r/2}, \quad R_{21} = \left( \frac{1}{2} \right)^{3/2} \frac{r}{\sqrt{3}} e^{-r/2}$$
$$Y_{00} = \sqrt{\frac{1}{4\pi}}, \quad Y_{10}(\theta, \varphi) = \sqrt{\frac{3}{4\pi}} \cos \theta, \quad Y_{1,\pm1}(\theta, \varphi) = \mp \sqrt{\frac{3}{8\pi}} \sin \theta e^{\pm i\varphi}$$
(1)算出 $n = 2$ 的所有矩阵元 $\langle \psi_{nlm} | z | \psi_{100} \rangle$;(2)结合你的计算结果,讨论这种矩阵元所涉及的有关 $\Delta m$ 与 $\Delta l$ 的选择定则(不考虑自旋)。

\subsection{5.52}
一维谐振子的能量本征态为 $|n\rangle, \hat{H}_0 |n\rangle = \left( n + \frac{1}{2} \right) \hbar \omega |n\rangle, n = 0, 1, 2, \cdots$。设有一微扰 $\hat{H}'$,满足 $\langle m | \hat{H}' | n \rangle = \begin{cases} \lambda, & n^2 + m^2 = 1 \\ 0, & \text{其他情况} \end{cases}$。体系的哈密顿量为 $\hat{H} = \hat{H}_0 + \hat{H}'$。如果 $t = 0$ 时体系处于基态,求 $t > 0$ 时体系处于各个态 $|n\rangle$ 上的概率。

\subsection{5.53}
当 $t < 0$ 时,质量为 $\mu$,角频率为 $\omega$,沿 x 方向振动的一维谐振子处于基态。从 $t > 0$ 时起,该振子受到沿 x 方向的力(不是势) $F(t) = F_0 e^{-t/\tau}$ 的作用,其中 $F_0$ 和 $\tau$ 是正的实数。假若 $F_0$ 很小,利用含时微扰论,准至一阶,求出振子在充分长时间后,处于各激发态的概率。

\subsection{5.54}
利用能量时间不确定关系,估算正负电子对能发生湮没相距的最大距离。