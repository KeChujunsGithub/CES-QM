\section{第六章}

\subsection{6.1}
电子偶素 ($e^+ e^-$ 束缚态)类似于氢原子,只是用一个正电子代替质子作为核。在非相对论近似下,其能量和波函数同氢原子相似。今设在电子偶素的基态 (s态)里,存在一种接触型自旋交换作用 $\hat{H}' = -\frac{8\pi}{3}\hat{M}_p \cdot \hat{M}_c s^3(r)$, 其中 $\hat{M}_p = \frac{e}{mc}\hat{S}_p$ 与 $\hat{M}_c = -\frac{e}{mc}\hat{S}_c$ 分别是正负电子的自旋磁矩。利用一级微扰论计算此基态中自旋单态与三重态之间的能量差,决定哪一能量更低。已知氢原子基态波函数

$$\psi_{100}(r) = \sqrt{\frac{1}{\pi a^3}} e^{-r/a}, \quad a = \frac{\hbar^2}{\mu e^2}, \quad \frac{e^2}{\hbar c} = \frac{1}{137}$$

\subsection{6.2}
电子偶素是由正负电子构成的类氢原子体系。考虑处于基态的电子偶素 ($l = 0$),系统的哈密顿量可写成 $\hat{H} = \hat{H}_0 + \hat{H}_s$,其中 $\hat{H}_0$ 是通常的与自旋无关的库仑力部分,$\hat{H}_s = A S_p \cdot S_e$ 是正电子与负电子的自旋作用部分。请问在无外磁场作用时,选择怎样的自旋和角动量本征态最方便?对这些态,计算由于 $\hat{H}_s$ 引起的能量的改变。

\subsection{6.3}
均匀磁场中电子偶素(电子-正电子束缚态)的哈密顿量(不考虑空间运动)为 $\hat{H} = J \sigma_1 \cdot \sigma_2 + FB (\sigma_{12} - \sigma_{22})$,其中 $J$ 与 $F$ 是常实数,$B$ 是磁场强度,下标 1 与 2 分别代表电子与正电子,$\sigma$ 是泡利矩阵。(1) 当 $B = 0$ 时,$\hat{H}$ 的本征函数与本征值是什么?(2) 当 $B \neq 0$ 时,$\hat{H}$ 的本征函数与本征值是什么?

\subsection{6.4}
一束极化的 s 波 ($l = 0$) 中子通过一个不均匀的磁场后分裂成强度不同的两束,其中自旋反平行于磁场的一束与自旋平行于磁场的一束的强度之比为 3:1。求入射中子自旋方向与磁场方向夹角的大小。

\subsection{6.5}
$\sigma_1, \sigma_2$ 为泡利矩阵,证明 $e^{i\alpha\sigma_j} = \cos \alpha + i\sigma_j \sin \alpha$,$j = 1, 2$,$\alpha$ 为实数并推广到矩阵 $\sigma = \lambda \sigma_1 + \mu \sigma_2 (\lambda^2 + \mu^2 = 1)$ 的情形。

\subsection{6.6}
有一定域电子(作为近似模型, 可以不考虑轨道运动)受到均匀磁场 $B$ 作用, 磁场指向 x 轴正方向, 相互作用势 $\hat{H} = \frac{eB}{\mu c}\hat{S}_x$. 设 t = 0 时电子自旋方向朝上, 即 $s_z = h/2$, 求 t > 0 时自旋 $\hat{S}$ 的平均值.

\subsection{6.7}
自旋 $s = 1/2$ 的电子在恒定磁场 $B = Bk$ 中运动。考虑电子自旋,求哈密顿算符 $\hat{H}$ 的表达式,求解 $\hat{H}$ 的本征值问题。如有简并,说明简并情况。

\subsection{6.8}
自旋 $s = 1/2$,并具有自旋磁矩 $\hat{M} = \mu_0 S$ 的粒子处于沿 $x$ 方向的均匀磁场 $B$ 中。已知 $t = 0$ 时,粒子的 $s_z = \hbar / 2$,求在以后任意时刻发现粒子具有 $s_y = \pm \hbar / 2$ 的概率。

\subsection{6.9}
给定 $(\theta, \varphi)$ 方向单位矢量 $n = (\sin \theta \cos \varphi, \sin \theta \sin \varphi, \cos \theta)$。在 $\sigma_z$ 表象,求 $\sigma_n = \sigma \cdot n$ 的本征值和归一化本征态,其中 $\sigma(\sigma_x, \sigma_y, \sigma_z)$ 为 3 个 2×2 的泡利矩阵。

\subsection{6.10}
自旋投影算符 $\hat{S}_n = \frac{\hbar}{2} \sigma \cdot n$,$\sigma$ 为泡利矩阵,$n$ 为单位方向矢量,$n^2 = 1$。

(1)对电子自旋朝上态 $\chi_+ (s_z = \hbar / 2)$,求 $\hat{S}_n$ 的可能值及相应概率。(2)对 $\sigma_n$ 的本征值为 1 的本征态,求 $\sigma_y$ 的可能值及相应的概率。

\subsection{6.11}
测量一个电子(处于自由空间)自旋 z 分量,结果为 $s_z = \hbar / 2$。(1)紧接着测量自旋 x 分量,得到的可能值与相应的概率是什么?(2)如果测量的自旋方向同 z 轴成 $\theta$ 角,得到的可能值与相应的概率是什么?期望值是什么?

\subsection{6.12}
$\sigma_x, \sigma_y, \sigma_z$ 为泡利矩阵,计算 $ e^{i\lambda \sigma_x} \sigma_x e^{-i\lambda \sigma_z} = ?$

\subsection{6.13}
$\sigma_x, \sigma_y, \sigma_z$ 为泡利矩阵,定义 $\sigma_x = \sigma_x \pm i\sigma_y$. (1) 计算 $[\sigma_x, \sigma_x]$, $[\sigma_z, \sigma_x]$, $[\sigma_z, \sigma_x](\sigma_+)^2, (\sigma_-)^2$; (2) 证明($\xi$ 为常数): $e^{\xi \sigma_z} \sigma_{\pm} = \sigma_{\pm} e^{\xi \sigma_z} e^{\pm 2 \xi}$; (3) 化简下面两式: $e^{\xi \sigma_z} \sigma_x e^{-\xi \sigma_x}, e^{\xi \sigma_x} \sigma_y e^{-\xi \sigma_y}$.

\subsection{6.14}
一束速度为 $v$ 自旋 $s = 1/2$ 在 z 轴方向极化 ($s_z = h/2$) 的中性粒子,沿 x 轴方向通过宽为 L 的均匀磁场区,磁场 B 的方向为正 x 轴方向。已知粒子具有自旋磁矩 $M = g\delta$,g 为常数。(1)求出粒子通过磁场区后其中 $s_z = -h/2$ 与 $s_z = h/2$ 的粒子数目之比;(2)如果希望通过磁场后的粒子全部都是 $s_z = -h/2$ 的,磁场强度 B 应取什么值?

\subsection{6.15}
把一个自旋 $s = 1/2$ 的粒子置于磁场 $B = B_0 (\sin \theta i + \cos \theta k)$ 中,其中 $i$ 与 $k$ 为 x 轴与 z 轴方向的单位矢量。体系的哈密顿量为 $\hat{H} = -\hat{M}_s \cdot B$,$\hat{M}_s = 2 \mu_B \hat{S}$ 是粒子的自旋磁矩。$\mu_B$ 是玻尔磁子。试求 $\hat{H}$ 的本征值与本征态矢。

\subsection{6.16}
已知氢原子哈密顿算符 $\hat{H} = \frac{\hat{p}^2}{2\mu} - \frac{e^2}{r} + \xi(r) \hat{L} \cdot \hat{S}$,在 $r - s_z$ 表象写出氢原子的定态方程。设此方程的某一定态解已经求出,它的归一化波函数为 $\psi (r, s_z, t) = \begin{pmatrix} \varphi (r) \\ F(r) \end{pmatrix} e^{-iEt/\hbar}$,其中 $\varphi$ 与 $F$ 是 $r$ 的已知函数,写出任意时刻力学量 $x$ 与 $s_x$ 的平均值计算公式。

\subsection{6.17}
氢原子基态能量 $E_1 = -e^2 / 2a$,其中 $a = h^2 / \mu e^2$ 为玻尔半径,$\mu$ 为折合质量,近似等于电子质量 $m_e$。(1)写出电子偶素(氢原子中质子由正电子代替)的基态能量和玻尔半径。(2)由于电子有自旋,电子偶素基态的简并度是多少? (3)电子偶素的基态会发生衰变,湮没为光子。这个过程中释放的能量和角动量是多少? 证明终态至少有 2 个光子。

\subsection{6.18}
考虑电子自旋谐振(自旋共振)。略去原子场,强磁场下 $B = B_0 k \cdot t = 0$ 时,电子自旋向上,并附加弱磁场 $B_1 = B_1 \sin \omega t \cdot t$。k 与 i 分别是 z 轴与 x 轴方向的单位矢量。试用一级含时微扰论,证明电子向自旋向下态的共振跃迁发生在 $\omega = \omega_0 = eB_0 / \mu c$ 处。略去非共振项,计算小 t 时在 $\omega = \omega_0$ 处电子到自旋向下态跃迁的概率。

\subsection{6.19}
自旋为1的带电粒子(电荷为$-q$,质量为$\mu$)受到磁场$B = Bj$的作用,其哈密顿量为 $\hat{H} = \frac{qB}{\mu c} \hat{S}_y$。如果$t = 0$时,粒子的自旋指向正x轴方向,求粒子自旋平均值的时间演化。

\subsection{6.20}
体系由两个自旋 $s = 1/2$ 的非全同粒子组成。已知粒子 1 处于 $s_{1z} = 1/2$ 的态上,粒子 2 处于 $s_{2x} = 1/2$ 的态上,求体系总自旋 $\hat{s}^2$ 与 $\hat{s}_z$ 的可测值及相应概率。

\subsection{6.21}
体系由两个自旋 $s = 1/2$ 的非全同粒子组成,粒子之间的相互作用为 $A\hat{S}_1 \cdot \hat{S}_2$,其中 $A$ 为常数。设 $t = 0$ 时,粒子 1 的自旋指向 z 轴正方向,粒子 2 的自旋指向 z 轴负方向。(1)在任意时刻测量粒子 1 的自旋处于 z 轴正方向的概率是多少?(2)在任意时刻测量粒子 1 与 2 的自旋处于 z 轴正方向的概率是多少?

\subsection{6.22}
一个体系由两个自旋 $s = 1/2$ 的非全同粒子组成,$\hat{S}_1$ 与 $\hat{S}_2$ 是粒子 1 与 2 的自旋算符。(1) 用非耦合态 $\alpha(1)\alpha(2), \alpha(1)\beta(2), \alpha(2)\beta(1), \beta(1)\beta(2)$,构成总自旋 $\hat{S} = \hat{S}_1 + \hat{S}_2$ 的 $\hat{S}^2$ 及 $\hat{S}_z$ 的共同本征态矢 $|sm_s\rangle$;(2) 求 $(\hat{S}_{1z} - \hat{S}_{2z})|sm_s\rangle = ?$ (3) 如体系的哈密顿量 $\hat{H} = A\hat{S}_1 \cdot \hat{S}_2 + B(\hat{S}_{1z} - \hat{S}_{2z})$,其中 $A$ 与 $B$ 为常数,求体系的能量;(4) 给出 $A = 0, B \neq 0$ 时 $\hat{H}$ 的归一化本征态。

\subsection{6.23}
两个自旋 $s=1/2$ 的非全同粒子体系哈氏量 $\hat{H} = V_0 (\sigma_{1x}\sigma_{2y}-\sigma_{1y}\sigma_{2x})$,其中 $V_0$ 是常数。(1)求 $\hat{H}$ 的本征值与本征态;(2)设 $t=0$ 时,体系处于 $\alpha(1)\beta(2)$ 态,求任意时刻体系处于 $\beta(1)\alpha(2)$ 态的概率。

\subsection{6.24}
3个自旋 s = 1/2 的粒子组成的体系哈密顿量为 $\hat{H} = C(\hat{S}_1 \cdot \hat{S}_2 + \hat{S}_2 \cdot \hat{S}_3 + \hat{S}_3 \cdot \hat{S}_1)$,其中 $C$ 为常数。求 $\hat{H}$ 的本征值与简并度。

\subsection{6.25}
一个两能级系统,哈密顿量为 $\hat{H_0}$,能级间隔大小为 $A$。现在此系统受到一微扰 $\hat{H}'$ 的作用,在 $\hat{H_0}$ 表象中 $\hat{H}'$ 的表示为 $\hat{H}' = \lambda (\sigma_1 + \sigma_2)$,其中 $\sigma_1$ 与 $\sigma_2$ 是泡利矩阵,$\lambda$ 为实数。请算出系统受微扰后能级的间隔。

\subsection{6.26}
试求在磁场强度为 $B$ 的外磁场中,电子的自由旋引起的能量本征值和本征函数,$B = B_z k + B_x i$, 其中 $B_x, B_z$ 是常数,$i$ 与 $k$ 分别是 $x$ 与 $z$ 方向的单位矢量。

\subsection{6.27}
电子在周期性变化的磁场中运动,$B_x = B_0 \cos \omega t, B_y = B_0 \sin \omega t, B_z = 0$。不考虑空间运动。已知 $t = 0$ 时,电子处于 $s_z = \hbar / 2$ 的态上,求任意 $t$ 时电子的波函数 $\psi(s_z, t)$,及电子处于 $s_z = -\hbar / 2$ 态的概率。

\subsection{6.28}
电子在周期性变化的磁场中运动, $B_x = B_0 \cos \omega t$, $B_y = B_0 \sin \omega t$, $B_z = B$. 不考虑空间运动。已知 $t = 0$ 时, 电子处于 $s_z = h/2$ 的态上, 求任意 $t$ 时电子的波函数 $\psi(s_z,t)$, 及电子处于 $s_z = -h/2$ 态的概率。

\subsection{6.29}
一束处于基态的氢原子通过 Stern-Gerlach 实验的不均匀磁场后分裂为两束。这两束氢原子中电子的自旋在磁场方向上的分量分别为 $h/2$ 与 $-h/2$,即氢原子中的电子被完全极化了。如果改用电子束重复上述实验,则电子束不能分裂为两束,为什么?

\subsection{6.30}
两个自旋为1/2的粒子处于态 $|\psi\rangle = a\alpha(2)\beta(1)+b\alpha(1)\beta(2)$ (纠缠态).求 (1)两个粒子自旋平行的概率;(2)两个粒子自旋反平行的概率;(3)体系处于总自旋为0的概率;(4)粒子1自旋向上的概率。当发现粒子1自旋向上时,粒子2处于什么态?

\subsection{6.31}
电子处于自旋 $\hat{S}$ 在方向 $n=(\sin\theta\cos\varphi,\sin\theta\sin\varphi,\cos\theta)$ 上投影 $\hat{S}\cdot n$ 的本征态,本征值为 $\hbar/2$。(1)求出相应的本征函数;(2)若在上面的态中,自旋的 x 分量和 y 分量有相等的均方差,请求出方向角 $\theta,\varphi$。

\subsection{6.32}
自旋为 $1/2$ 的粒子具有自旋磁矩 $\hat{M}_s = \gamma \hat{S}$,该粒子处于磁场 $B = Bn(\theta, \varphi)$ 中,$n(\theta, \varphi)$ 是 $(\theta, \varphi)$ 方向的单位矢量。设 $t = 0$ 时粒子处于自旋朝下态 $|\psi(0)\rangle = |-\rangle$,求 t 时刻粒子仍处于该态的概率。

\subsection{6.33}
求哈密顿量 $\hat{H} = \sigma_{1x} \sigma_{2x} + \sigma_{1y} \sigma_{2y} + a \sigma_{1z} \sigma_{2z}$ 的本征值与本征态,其中 $\sigma_{1x}, \sigma_{2y}, \cdots$ 是粒子1与2的泡利矩阵的 $x, y$ 分量等,$a$ 为实数,并讨论 $a=1$ 时的特点。

\subsection{6.34}
讨论一个由电子和正电子通过库仑吸引力结合而成的类氢原子体系。该体系在磁场 $B = Be_z$ 中的哈密顿量为 $\hat{H} = \hat{H}_0 + A\hat{S_e} \cdot \hat{S_p} + \frac{eB}{mc} (\hat{S_{ez}} - \hat{S_{pz}})$,式中 $\hat{H}_0$ 是电子与正电子的动能,以及电子与正电子之间的库仑能之和,$\hat{S_e}, \hat{S_p}$ 分别为电子和正电子的自旋,取 $A\hat{S_e} \cdot \hat{S_p} + \frac{eB}{mc} (\hat{S_{ez}} - \hat{S_{pz}})$ 为微扰,用微扰论求 $\hat{H}_0$ 由于自旋而导致的四度简并的基态($l=0$)能量变化至一级修正。

\subsection{6.35}
一个由三个非全同的自旋为 $1/2$ 的粒子组成的体系哈密顿量为 $\hat{H} = \frac{A}{h^2}\hat{S}_1 \cdot \hat{S}_2 + \frac{B}{h^2}(\hat{S}_1 + \hat{S}_2) \cdot \hat{S}_3$,其中 $\hat{S}_1, \hat{S}_2, \hat{S}_3$ 分别是三个粒子的自旋算符,求体系的能量与相应的简并度。

\subsection{6.36}
考虑二维电子系统中存在自旋-轨道耦合,$\hat{H}=\hat{H}_0+\hat{H}_{s.o}$, $\hat{H}_0$是二维自由电子哈密顿量:$\hat{H}_0=\frac{\hat{p}^2}{2m}=\frac{\hat{p}_x^2}{2m}+\frac{\hat{p}_y^2}{2m}$。电子在xy平面中运动。$\hat{H}_{s.o}$表示电子自旋与轨道的耦合作用:$\hat{H}_{s.o}=\frac{\lambda}{h}(\hat{p}_y\hat{\sigma}_x-\hat{p}_x\hat{\sigma}_y)$。假设电子波函数可表示为自旋波函数与轨道运动波函数的直积形式:$\psi(r,s_z)=\psi(r)\chi(s_z)$。(1)对哈密顿量$\hat{H}_0$,求解本征值问题,并说明对$s_z=\pm h/2$,能量是简并的;(2)对哈密顿量$\hat{H}=\hat{H}_0+\hat{H}_{s.o}$,求出本征值及相应的本征函数。

\subsection{6.37}
两个自旋为 $1/2$ 的非全同粒子体系.以 $+,-$ 分别代表自旋向上与向下的两个态.在 $t=0$ 时体系波函数为 $|\psi(0)\rangle = \frac{1}{2}|++\rangle + \frac{1}{\sqrt{2}}|+-\rangle + \frac{1}{2}|--\rangle$.体系的哈密顿量为 $\hat{H} = \omega_1 \hat{S}_{1z} + \omega_2 \hat{S}_{2z}$.(1)求 $t$ 时刻波函数 $|\psi(t)\rangle$;(2)求 $t$ 时刻的平均值:$\langle s_{1x}\rangle$, $\langle s_{1y}\rangle$, $\langle s_{2x}\rangle$ 与 $\langle s_{2y}\rangle$.

\subsection{6.38}
自旋均为 $1/2$ 的两个非全同粒子构成孤立体系,粒子间存在相互作用 $\alpha (S_{1x} S_{2x} + S_{1y} S_{2y} - S_{1z} S_{2z})$,其中 $\alpha$ 为实常数。只考虑自旋自由度,(1)求体系的能量本征值问题,并讨论能量的简并度;(2)设 $t=0$ 时,粒子 1 与 2 的自旋分别沿 z 轴正向与 x 轴负向,求 $t>0$ 时,粒子 1 自旋反转的概率。

\subsection{6.39}
磁矩为 $\hat{\mu} = -\gamma \hat{S}$ 的电子在恒定磁场 $B = Be_y$ 中运动 ($\gamma, B$ 均为正实数)。初始时刻电子处于 $s_z = -\hbar / 2$ 的态上,求 (1) $t > 0$ 时 $\hat{S}_y$ 与 $\hat{S}_z$ 的平均值;(2) 电子自旋 x 分量反转周期(由 $s_x = \hbar / 2 \rightarrow s_x = -\hbar / 2$ 的时间)。

\subsection{6.40}
磁矩为 $\hat{M}_s = -\gamma S$ 的电子在外磁场 $B = Be_z$ 中运动,现加上另一外磁场 $B' = B'e_x (\gamma > 0, B, B'$ 均为实数)。(1)严格求解该电子的能量本征值;(2)若将 $B'$ 的作用视为微扰,求电子能量本征值至二级近似,本征态至一级近似。

\subsection{6.41}
(1)考虑自旋为1/2的系统。试在 $\hat{S}^2, \hat{S}_z$ 表象中求算符 $A\hat{S}_y + B\hat{S}_z$ 的本征值及归一化的本征态,其中 $\hat{S}_y, \hat{S}_z$ 是自旋角动量算符,而 $A, B$ 为实常数。(2)假定此系统处于以上算符的一个本征态上,求测量 $\hat{S}_y$ 得到结果为 $h/2$ 的概率。

\subsection{6.42}
能量为 $E$ 的中子束沿 x 轴入射,在 $x>0$ 区受到势场 $V(x)=V_0+a_{0}\hat{S}_z$ 的作用,其中 $V_0,a_{0}$ 是正实数,$\hat{S}_z$ 是中子自旋 z 分量,且 $0<\hbar a_{0}/2<V_0$。在 $x<0$ 处,$V(x)=0$。(1)设入射中子束中自旋 z 分量向上和向下的中子各占一半。求反射中子的极化度: $A= \text{(自旋向上中子数-自旋向下中子数)/中子总数}$ (2)入射中子能量为何值时,极化度最大?

\subsection{6.43}
有一个自旋为 $1/2$,磁矩为 $\mu$ 电荷为 0 的粒子,置于磁场 $B = B_0 k$ 中,$t = 0$ 时处于自旋“向下”态 ($\sigma_z = -1$),$t > 0$ 时再加上弱的磁场 $B = B_1 i$。求 $t > 0$ 时粒子的自旋态,以及测得自旋“向上” ($\sigma_z = 1$) 的概率。

\subsection{6.44}
证明 $\sqrt{1+\sigma_x}$ 是厄米算符,其中 $\sigma_x$ 是泡利矩阵。

\subsection{6.45}
电子的磁矩定义为 $\hat{M} = \hat{M}_L + \hat{M}_S = -\frac{e}{2\mu c}\left(\hat{L} + 2\hat{S}\right)$。计算它的 $z$ 分量 $\hat{M}_z = -\frac{e}{2\mu c}\left(\hat{L}_z + 2\hat{S}_z\right)$ 在 $\hat{L}^2, \hat{S}^2, \hat{J}^2, \hat{J}_z$ 的共同本征态 $|lsjm_j\rangle$ 上的平均值 $\langle M_z|_{lsjm_j}$,其中 $m_j = j$ 的 $\langle M_z|_{lsjj} = \mu$ 用来表示电子磁矩的大小,算出 $\mu$。将电子的磁矩的计算结果,推广到原子磁矩。[提示:$j = l \pm (1/2)$ 的 $|lsjm_j\rangle$ 表示式为 $l_s s, j = l + \frac{1}{2}, m_j = \sqrt{\frac{l + m_j + \frac{1}{2}}{2l + 1}} \left| l,m_j - \frac{1}{2} \right\rangle \alpha + \sqrt{\frac{l - m_j + \frac{1}{2}}{2l + 1}} \left| l,m_j + \frac{1}{2} \right\rangle \beta$ 和 $l_s s, j = l - \frac{1}{2}, m_j = \sqrt{\frac{l - m_j + \frac{1}{2}}{2l + 1}} \left| l,m_j - \frac{1}{2} \right\rangle \alpha + \sqrt{\frac{l + m_j + \frac{1}{2}}{2l + 1}} \left| l,m_j + \frac{1}{2} \right\rangle \beta$ 其中 $s = 1/2$, $|lm\rangle$ 是 $\hat{L}^2$ 与 $\hat{L}_z$ 的共同本征态。]

\subsection{6.46}
在电子的某个自旋态 $|\chi\rangle$ 中,测量 $\hat{S}_x$ 得 $h/2$ 的概率为 1/6,测量 $\hat{S}_y$ 得 $h/2$ 的概率为 1/3。求该自旋态 $|\chi\rangle$ 和 $\hat{S}_x$ 的平均值。

\subsection{6.47}
1/2 自旋算符可用泡利矩阵 $\sigma = (\sigma_1, \sigma_2, \sigma_3)$ 表示为 $\hat{S} = h \sigma / 2$,其中 $\sigma_3$ 的两个本征态为 $\alpha$ 与 $\beta$。(1) 求 $\sigma \cdot n$ 的本征态,其中 $n = (\sin \theta \cos \varphi, \sin \theta \sin \varphi, \cos \theta)$;(2) 求算符 $\hat{U} = e^{-i \sigma_3 \varphi / 2} e^{-i \sigma_2 \theta / 2}$ 对 $\sigma_3$ 的两个本征态 $\alpha$ 与 $\beta$ 作用的结果;(3) 说明(1) 和(2) 题结果之间的关系及 $\hat{U}$ 的物理意义。

\subsection{6.48}
一个质量为 $m$ ,无电荷但自旋为 $1/2$,磁矩为 $\hat{M}_s = -\frac{\mu_0}{2} \hat{S}$ 的粒子在一维无限深势阱 $V(x) = \begin{cases} 0, & |x| < L \\ \infty, & |x| > L \end{cases}$ 中运动,其中 $\mu_0$ 和 $L$ 是正的常数, $\hat{S}$ 为粒子的自旋算符.现考虑在 $x < 0$ 的半空间中有一沿 $z$ 方向的均匀磁场,大小为 $B$ ,而在 $x > 0$ 的半空间中有一同样大小但沿 $x$ 方向的均匀磁场.在弱磁场极限下用微扰论求出体系基态的能级和波函数,并指出 $B$ 能作为弱磁场处理的具体条件[微扰只需计算到最低阶,自旋空间的波函数在 $(\hat{S}^2, \hat{S}_z)$ 表象中写出].

\subsection{6.49}
电子在磁场 $B = (0, B_y, B_z)$ 中运动,$B_z$ 比 $B_y$ 小得多。(1)只考虑自旋运动,写出体系的哈密顿量 $\hat{H}$,求出本征能量并展开到 $B_z / B_y$ 的领头阶,求出 $\hat{H}$ 的本征态;(2)把 $B_z$ 看成微扰,用定态微扰论计算体系的能量到 $B_z^2$ 阶。

\subsection{6.50}
氢原子处于沿 z 轴方向的均匀电场与磁场中,电场 $E = \varepsilon k$,磁场 $B = Bk$ 如果电场和磁场足够强,以致可以忽略自旋-轨道耦合作用,而电磁作用仍可当做微扰,求计入电子自旋后氢原子 $n = 2$ 能级的分裂情况(一级近似)。

\subsection{6.51}
已知电子某一时刻的波函数为 $\psi(r,t) = \psi_1(r) \chi_+(s_z) + \psi_2(r) \chi_-(s_z)$,其中 $\chi_+$ 与 $\chi_-$ 分别是电子自旋沿正 z 轴与负 z 轴的归一化波函数。试写出电子自旋沿正 z 轴且在薄壳 $(r,r+dr)$ 内出现的概率 $P_1$;电子自旋沿负 z 轴且在 $(\theta,\varphi)$ 方向 $d\Omega = \sin\theta d\theta d\varphi$ 立体角内出现的概率 $P_2$;电子自旋沿正 z 轴的概率 $P_3$。

\subsection{6.52}
一个自旋 $s = 1/2$ 的粒子,其哈密顿量为 $\hat{H} = \frac{\hbar \omega}{5}(3\sigma_z + 4\sigma_x)$,其中 $\omega$ 为常量,$\sigma_z$ 与 $\sigma_x$ 为泡利矩阵。(1)求粒子的能级与相应的波函数;(2)已知 $t = 0$ 时粒子的自旋沿正 z 轴,求 $t > 0$ 时粒子的波函数 $|\psi(t)\rangle$,粒子沿负 z 轴的概率 $P$,以及能量的平均值 $\bar{E}$。

\subsection{6.53}
(1)证明 $\hat{L}^2 = \sigma \cdot \hat{L}(\sigma \cdot \hat{L} + \hbar)$,其中 $\hat{L}$ 是轨道角动量,$\sigma$ 为泡利矩阵;(2)计算 $(\sigma \cdot \hat{L})^2$ 在 $\hat{L}^2$、$\hat{J}^2$ 与 $\hat{J}_z$ 的共同本征态 $|\hat{J}m_j\rangle$ 态上的平均值,其中 $\hat{J} = \hat{L} + \hat{S}$ 是总角动量,$\hat{S} = \hbar \sigma / 2$ 是自旋。

\subsection{6.54}
证明如果在电子的某一态上测量自旋 x 分量和 y 分量的平均值均为 0,则测量自旋 z 分量时,不是 $h/2$,就是 $-h/2$。

\subsection{6.55}
自旋 $s=1/2$ 的核子处于三维各向同性谐振子势场 $V(r) = \frac{1}{2} \mu \omega^2 r^2$ 中,定态波函数与能量为 $\psi_{n_r \text{Imm}_s} = \psi_{n_r \text{Im}}(r)\chi_{m_s}(s_z) = R_{n_r}(r)Y_{\text{Im}}(\theta,\varphi)\chi_{m_s}(s_z), E_N = \left( N + \frac{3}{2} \right) h \omega$ $n_r,l=0,1,2,\cdots,m=0,\pm1,\cdots,\pm l,m_s=\pm1/2,N=2n_r+l=0,1,2,\cdots$ (1)指出 $N=2$ 的能级 $E_2$ 的简并度;(2)如果核子还受到 $\hat{H}' = -\epsilon \hat{L} \cdot \hat{S}$ (c为正实数)的自旋-轨道耦合作用,讨论 $N=2$ 的能级 $E_2$ 的分裂情况,并指出分裂后能级的简并度。

\subsection{6.56}
质量为 $m$,能量 $E > 0$,自旋向上(沿 z 轴方向)的电子沿 x 轴从 $-\infty$ 向右运动,在 $x = 0$ 处被位势 $V(x, \sigma) = -A \delta (x)(\sigma_x + \sigma_y)$ 散射,其中 $\sigma_x$ 与 $\sigma_y$ 为电子自旋的升降算符,$A > 0$。求散射后电子具有自旋向下(沿 z 轴反方向)的反射份额。
